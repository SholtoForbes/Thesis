% 3_methodology.tex

\cleardoublepage
\chapter{Launch Vehicle Baseline Design}\label{chapter:methodology}

This chapter presents the baseline three stage small satellite launch system utilised in this thesis. The design for the first stage rocket-powered vehicle, second stage scramjet-powered vehicle (the SPARTAN), and third stage rocket-powered vehicle are presented. This baseline design is used for the initial trajectory analysis and optimisation in this study. The launch system has been designed based on the SPARTAN vehicle developed by Preller \& Smart.

This chapter also describes the aerodynamic models of the vehicles used in this study, including the engine models.


\begin{figure}
\centering
\includegraphics[width=0.7\linewidth]{figures/3_vehicle_design/NoInternal}
\caption{}
\label{fig:NoInternal}
\end{figure}

\begin{figure}
\centering
\includegraphics[width=0.7\linewidth]{figures/3_vehicle_design/INTERNALS}
\caption{}
\label{fig:INTERNALS}
\end{figure}


\section{First Stage Rocket}
The first stage rocket-powered vehicle is based on the first stage of the SpaceX Falcon-1e. The Falcon-1e has been chosen due to its appropriate scale, and the proven flightworthiness of the Falcon-1.

	detailed design
	First Stage CART3D
	- details of CAD, pointwise mesh, and CART3D gridding settings
	
	
	\section{Second Stage Scramjet}
		\subsection{Baseline Vehicle}
		The SPARTAN vehicle in this study has been designed based on the work by Preller \& Smart CITATION. 
		
		The baseline SPARTAN has been designed to hold a 9 m long third stage.  Two cylindrical tanks underneath the third stage and a conical tank situated in the nose have a total volume of 22.0m3 and hold a total of 1562kg of LH2 fuel. This assumes an LH2 density of 71kg/m3, slightly denser than LH2 at phase transition point at 1 atm.
		
		 %http://webbook.nist.gov/cgi/fluid.cgi?Action=Load&ID=C1333740&Type=SatT&Digits=5&PLow=.5&PHigh=1.5&PInc=.1&RefState=DEF&TUnit=K&PUnit=atm&DUnit=kg/m3&HUnit=kJ/mol&WUnit=m/s&VisUnit=uPa*s&STUnit=N/m%
		
		
		
		
		
		
		fuel tanks are sized to include a kestrel third stage
		
		total tank volume = 22m$^3$
		
		gives a fuel mass of 1562kg
		
		
		
		Dawid has a fuel tank volume of 14.01m$^3$, mass of 132.8kg
		
		using 22$^2/3$  /  14.01$^2/3$  to approximate surface area ratio
		
		mass of new tank is 1.351*132.8 = 179.41kg
		
		
		CG - determined using CREO
		
		\subsection{Aerodynamics}
		method of calculating aerodynamics for trim at required lift
		
		conical shock calculation
		
		\begin{figure}
\centering
\includegraphics[width=0.7\linewidth]{figures/3_vehicle_design/CARTmesh}
\caption{}
\label{fig:CARTmesh}
\end{figure}


CART3D lift/drag 

		
		\subsection{Propulsion Modelling}
		
		The SPARTAN is powered by four underslung scramjet engines. These engines are Rectangular To Elliptical Shape Transition (REST) engines, configured to allow for a conical forebody (C-REST). The engine model used is a CRESTM10, analysed using quasi-1D simulation. The specific impulse and equivalence ratio data sets are shown in Figure XXX. The specific impulse and equivalence ratio during flight are determined by interpolating these data sets. 
		As the equivalence ratio is equal to 1 in the majority of the Mach number and temperature regime, only the regions in which it is less than 1 are used for the interpolation. The equivalence ratio interpolation is linear, as the number of data points available for interpolation is low. 
	
	
	\section{Third Stage Rocket}
	
	\begin{figure}
\centering
\includegraphics[width=0.7\linewidth]{figures/3_vehicle_design/3rdStage}
\caption{}
\label{fig:3rdStage}
\end{figure}

	The third stage vehicle has been designed to accommodate a SpaceX Kestrel engine. The Kestrel has been chosen for its cost effectiveness and proven capabilities. 
	
	The third stage rocket is released at the end of the scramjet accelerator burn, and lifts the payload out of the atmosphere and into the desired orbit. The third stage weighs a total of 3300kg. The third stage has a structural mass fraction of 0.09, similar to the Falcon 1 second stage \cite{Vehicle2008}. This gives a total structural mass of 285.7kg. 
	
	

	
The kestrel engine has been modified to have 50\% increased propellant mass flow rate, giving a mass flow rate of 14.8kg/s. The nozzle of the Kestrel engine has been kept at XXXm diameter. This increase in mass flow results in a 2\% loss of efficiency from the nozzle (CITATION). The modified specific impulse of the engine is 310.7s.

	
	Area = 0.866
	
	Lnose = 3m, Centrebody = 4.5m, Engine = 1.5m. Note, some room has been left for the kestrel to go into the body slightly (about 0.5m)
	
	CG - determined using creo, assumed that structure is homogeneous for simplicity
		
		\subsection{Fuel Tank Sizing}
		The fuel tanks have been sized assuming 100kg of payload-to-orbit. Note that the method of calculating final payload-to-orbit relies on using left over 'fuel mass' as effective payload mass. Realistically this would cause the fuel tanks to be resized slightly. For the purposes of this study the fuel tanks have been assumed to be of constant size for simplicity. Currently this is a reasonable assumption as the internals of the rocket are very simplified. The structural mass is held constant at 9\%. 
		

		
		Propellant mass: 2736.7 kg.
		
		Propellant ratio is 2.56 LOX : 1 RP1
		
		mass LOX = 1968.0 kg.
		$\rho_{LOX}$ = 1141kg/m3.
		volume LOX = 1.7248m3
		
		
		mass RP1 = 768.7 kg.
		$\rho_{RP1}$ = 813kg/m3.
		http://nvlpubs.nist.gov/nistpubs/Legacy/IR/nistir6646.pdf
		
		volume rp1 = 0.9455m3
		
		
		
		
		
		
		\subsection{Aerodynamics}
		Third Stage missile DATCOM details
		
		\subsection{Thrust Vectoring}
		
		The third stage rocket is trimmed during the in-atmosphere portion of its ascent trajectory via thrust vectoring. The centre of pressure is calculated using missile DATCOM. The thrust vector is set so that the moment generated by the engine matches the lift force acting at the centre of pressure. The maximum thrust vector limit has been set to 8$^\circ$. As no data on the maximum thrust vectoring capabilities of the kestrel engine was able to be found, this was set to the maximum gimbal range of the Aestus pressure-fed engine and OMS, similar pressure fed engines. 
		
		
		\subsection{Heat Shield Sizing}
		
			The third stage is protected while in-atmosphere by a heat shield, weighting 130.9kg. This heat shield is constructed from a phenolic cork cylinder, a reinforced carbon-carbon nose cone, and a tungsten nose tip. 
		
		
		The tungsten nose is 50mm diameter, at the end of a 50mm Cylinder. The density of tungsten is $\rho_{Tungsten} = 19.25$  g/cm$^3$, giving a total mass for the nose of m = 12.6kg.
		
		Carbon-Carbon shell
		$\rho_{CC} = 1800$  kg/m$^3$
		Thickness of 10mm
		L = 3m
		R = 0.525
		Area = 5.0232m2 ( from Creo)
		
		Anose = 6.22e-2 m2
		
		m = (Area - Anose) t $\rho$ = 89.3kg
		
		
		Cork Cylinder
		t = 5mm
		$\rho_{Cork} = 320$  kg/m$^3$
		
		Area = 14.844m2
		
		m = 23.7kg
		
		
		$m_HS$ = 125.6kg
		