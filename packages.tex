% packages.tex

%\pdfoptionpdfminorversion=6

% Setting margins
\usepackage[left=20mm,right=20mm,top=20mm,bottom=20mm,headheight=15pt,includehead,includefoot]{geometry}

% For adding images
\usepackage{graphicx}
\graphicspath{{./figures/1_introduction/}
              {./figures/1_introduction/}
              {./figures/2_literature-review/}
              {./figures/3_methodology/}
              {./figures/4_experimental-results/}
              {./figures/5_numerical-results/}
              {./figures/6_conclusions}}

% For colors
\usepackage[svgnames,dvipsnames,x11names,table]{xcolor}

% Nomenclature
\usepackage[intoc]{nomencl}
\usepackage{ifthen}
\renewcommand\nomgroup[1]{%
  \ifthenelse{\equal{#1}{A}}{%
    \item[\textbf{Acronyms and Initialisms}]}{% A - Acronyms
  \ifthenelse{\equal{#1}{R}}{%
    \item[\textbf{Roman Symbols}]}{%            R - Roman
  \ifthenelse{\equal{#1}{G}}{%
    \item[\textbf{Greek Symbols}]}{%            G - Greek
  \ifthenelse{\equal{#1}{S}}{%
    \item[\textbf{Superscripts}]}{%             S - Superscripts
  \ifthenelse{\equal{#1}{U}}{%
    \item[\textbf{Subscripts}]}{%               U - Subscripts
  \ifthenelse{\equal{#1}{X}}{%
    \item[\textbf{Other Symbols}]}{%            X - Other Symbols
  {}}}}}}}}
\renewcommand*{\nompreamble}{\markboth{\nomname}{\nomname}}
\renewcommand{\eqdeclaration}[1]{, see Equation~(#1)}
\newcommand{\nomunit}[1]{%
  \renewcommand{\nomentryend}{\hspace*{\fill}#1}%
  }
\renewcommand{\nompreamble}{The nomenclature used in this thesis is outlined here. In some cases, the same variable has been used to indicate different quantities. In these instances, the equation and/or page number of the `special' case has been included.}
%\setlength{\nomlabelwidth}{2.5cm}

% I think this was just a formatting thing to fix long nomenclature entries
% \newcommand{\nomunit}[1]{%
%   \renewcommand{\nomentryend}{\hspace*{\fill}\makebox[1cm][l]{#1}}%
%   }

\makenomenclature

% For subfigures
\usepackage{subcaption}

% Font
\usepackage{mathptmx}
\usepackage{amsmath}
\DeclareMathOperator\erf{erf}

% For the header/footer (Edit these as you wish)
\usepackage{fancyhdr}
\pagestyle{fancy}
\fancyhead{}
\fancyhead[LE]{\leftmark}
\fancyhead[RO]{\rightmark}
\fancyfoot{}
\fancyfoot[LE,RO]{\thepage}

% Chapter heading style (Edit these as you wish)
\usepackage[Conny]{fncychap}
\ChNameVar{\raggedleft\Large\scshape}
\ChTitleVar{\Large\scshape}
\ChNumVar{\Huge}

% Clean empty pages of formatting
\usepackage{emptypage}

% Line spacing
\usepackage{setspace}
\onehalfspacing

% Tables
\usepackage{booktabs}
\usepackage{threeparttable}


% For diagrams
\usepackage{tikz}
\usetikzlibrary{arrows,decorations.pathmorphing,patterns}


% For bibliography
\usepackage[backend=bibtex]{biblatex}
\setcounter{biburlucpenalty}{9000}
\addbibresource{library.bib}
%
%\renewcommand*{\nameyeardelim}{\addcomma\space}
%\newcommand{\citeauthorandyear}[2][]{\citeauthor{#2} (\citeyear[#1]{#2})}

%\DeclareCiteCommand{\citelabelyear}
%  {\boolfalse{citetracker}%
%   \boolfalse{pagetracker}%
%   \usebibmacro{prenote}}
%  {\printtext[bibhyperlink]{\iffieldundef{labelyear}
%   {\printfield{year}}
%   {\printfield{labelyear}%
%    \iffieldundef{extrayear}
%      {}
%      {\printfield{extrayear}}}}}
%  {\multicitedelim}
%  {\usebibmacro{postnote}}
%
%\DeclareCiteCommand{\citeyear}
%    {}
%    {\bibhyperref{\printdate}}
%    {\multicitedelim}
%    {}

% For links
\PassOptionsToPackage{hyphens}{url}
\usepackage[]{hyperref}
\hypersetup{
    colorlinks,
    linkcolor={blue!80!black},
    citecolor={blue!80!black},
    urlcolor={blue!80!black}
}

% \usepackage[hidelinks]{hyperref} % Hide links

% For including PDF pages
\usepackage{pdfpages}
% For adding dummy text
\usepackage{lipsum}
% For trademark (TM) text
\usepackage{textcomp}
% Strikeout (line through text)
\usepackage{soul}
% multirow in tables
\usepackage{multirow}
% For including landscape pages
\usepackage{lscape}
% To scale content to fit pages
\usepackage{adjustbox}



% Remove spacing in itemize environment
\newenvironment{myitemize}
{ \begin{itemize}
    \setlength{\itemsep}{0pt}
    \setlength{\parskip}{0pt}
    \setlength{\parsep}{0pt}     }
{ \end{itemize}                  }

% For consistent units and number range formatting
\usepackage[space-before-unit = true]{siunitx}
%\sisetup{%
%  inter-unit-product=\ensuremath{{}\cdot{}},
%  range-units = single,
%  range-phrase =-,
%  group-separator = \text{,},
%  }
%\DeclareSIUnit[]\atm{atm}
\DeclareSIUnit[]\cal{cal}
    
% For controlling float positions
\usepackage{placeins}

% Forces something to do something. Can't quite remember
\interfootnotelinepenalty=10000


\usepackage{pdflscape}

\usepackage{tabularx}