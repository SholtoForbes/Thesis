% 1_introduction.tex

\cleardoublepage
\chapter{Introduction}

  
  
  The market for small satellites is increasing rapidly, driven primarily by the rapidly decreasing size of electronic components. Many of these small satellites are currently launched with multiple small satellites piggybacking onto the launch of larger satellites, comanifesting on a single launch to share costs. This has the result
  of smaller payloads being subject to the schedule of the other parties involved, and being delivered into
  an orbit determined by the requirements of the main payload. This dependence on a predetermined mission plan often has detrimental effects on the small satellite mission. Small satellites are often used in missions which require specific orbital and scheduling needs. Satellites to be used as part of constellations are especially sensitive to the orbit in which they are placed. Small satellites may be able to change orbits slightly, however larger changes mean additional fuel mass and potentially additional system mass if a larger drive is necessary. 
  The production of a cheap and flexible small satellite delivery system would enable more small satellites to be delivered into customised orbits within relatively short time periods.
  
  Airbreathing engines are an ideal candidate for producing the next generation of space access
  systems, producing higher specific impulse than rockets and requiring much less fuel to be carried
  enabling the design of a reusable second stage vehicle. A three stage design utilising a scramjet
  second stage with rocket powered first and third stages is being developed by The University of
  Queensland. In previous studies it has been assumed that the optimal trajectory for a scramjet powered vehicle is at its
  maximum dynamic pressure and all other trajectory stages have conformed to this assumption. This study will aim to produce an optimal
  trajectory plan which may be applied to any rocket-airbreathing-rocket system for delivering small
  satellites to Earth orbit. The SPARTAN vehicle being developed by the university of Queensland will
  be used as a model for simulations.
  
  \section{Research aims}

    The overall aim of this thesis is to

    \begin{quote}
      \emph{make scramjets fly}.
    \end{quote}

    \noindent This will be achieved through the following objectives:

    \begin{enumerate}
      \item \emph{Flying them}\\
      Flying scramjets will conclusively prove that scramjets can fly.

      \item \emph{Fly different types of scramjets}\\
      By flying lots of scramjets we can see that they are awesome. 

      \item \emph{Having 3 objectives is typically good.}\\
      Three objectives just makes this section look complete.
    \end{enumerate}

  \clearpage
  \section{Thesis outline}

    Edit this as required. This thesis is organised in eight chapters which are outlined below. The appendices included at the end of this document contain further technical information and supplementary experimental details. \textcolor{red}{If you want to add colours to the text for some reason (supervisor reviewing etc.), use the \texttt{textcolor} command. Check the code here to see how.}

    \subsubsection*{Chapter 2 - Literature review}

      This chapter presents a review of literature related to the different aspects of this thesis. 

    \subsubsection*{Chapter 3 - Methodology}

      This chapter provides details of the experimental facility, experimental model, and test conditions used in the completion of the experiments that form the majority of this thesis. 

    \subsubsection*{Conclusions}

      The body of this thesis concludes by summarising the most significant findings from Chapters~\ref{chapter:jpp} through~\ref{chapter:numerical}. Recommendations for future studies on cavity flameholders in three-dimensional, hydrocarbon-fuelled scramjet engines, such as the REST engine, are provided.
