% numerical.tex

\cleardoublepage
\chapter{Ascent Trajectory}\label{chapter:numerical}


- keep this DIDO and hypaero and direct shooting

-explain why I moved to other GPOPS and CART3D at the end of the chapter



\section{Third Stage trajectory}
-Details on methodology
-outline the limits that I put on it (ie cant go below starting altitude)
-I should maybe run a case without these limits to show the difference
-outline necessity for differing guess (will need to cite this as a regularly used process)

\begin{figure}
\centering
\includegraphics[width=0.7\linewidth]{figures/5_Ascent/ThirdStageConstQ}
\caption{}
\label{fig:ThirdStageConstQ}
\end{figure}
\begin{figure}
\centering
\includegraphics[width=0.7\linewidth]{figures/5_Ascent/ThirdStage50kpaconstrained}
\caption{}
\label{fig:ThirdStage50kpaconstrained}
\end{figure}



\section{SPARTAN trajectory}
-constant q
-45kPa,50kPa and 55kPa limited trajectories
-high drag trajectory

-it might be interesting to compare different third stage rocket engines 

\subsection{Constant Dynamic Pressure}
\begin{figure}
\centering
\includegraphics[width=0.7\linewidth]{figures/5_Ascent/Constq}
\caption{}
\label{fig:Constq}
\end{figure}
\begin{figure}
\centering
\includegraphics[width=0.7\linewidth]{figures/5_Ascent/Constq-Aero}
\caption{}
\label{fig:Constq-Aero}
\end{figure}
\begin{figure}
\centering
\includegraphics[width=0.7\linewidth]{figures/5_Ascent/Constq-Vehicle}
\caption{}
\label{fig:Constq-Vehicle}
\end{figure}

\subsubsection{Optimality Validation}


\subsection{Optimal Payload}
\begin{figure}
\centering
\includegraphics[width=0.7\linewidth]{figures/5_Ascent/qlimited50kpa}
\caption{}
\label{fig:qlimited50kpa}
\end{figure}
\begin{figure}
\centering
\includegraphics[width=0.7\linewidth]{figures/5_Ascent/qlimited50kpa-Aero}
\caption{}
\label{fig:qlimited50kpa-Aero}
\end{figure}
\begin{figure}
\centering
\includegraphics[width=0.7\linewidth]{figures/5_Ascent/qlimited-Vehicle}
\caption{}
\label{fig:qlimited-Vehicle}
\end{figure}

\subsection{Maximum Dynamic Pressure Variation}
\begin{figure}
\centering
\includegraphics[width=0.7\linewidth]{figures/5_Ascent/Multipleq}
\caption{}
\label{fig:Multipleq}
\end{figure}
\begin{figure}
\centering
\includegraphics[width=0.7\linewidth]{figures/5_Ascent/MultipleqAero}
\caption{}
\label{fig:MultipleqAero}
\end{figure}
\begin{figure}
\centering
\includegraphics[width=0.7\linewidth]{figures/5_Ascent/Multipleq-Vehicle}
\caption{}
\label{fig:Multipleq-Vehicle}
\end{figure}

\subsection{Additional Drag Design Study}
\begin{figure}
\centering
\includegraphics[width=0.7\linewidth]{figures/5_Ascent/DragComparisonTraj}
\caption{}
\label{fig:DragComparisonTraj}
\end{figure}
\begin{figure}
\centering
\includegraphics[width=0.7\linewidth]{figures/5_Ascent/DragComparisonOther}
\caption{}
\label{fig:DragComparisonOther}
\end{figure}

\section{First Stage Trajectory}
multiple release points
-fixed mass, optimal velocity
-fixed velocity, optimal mass


\begin{figure}
\centering
\includegraphics[width=0.7\linewidth]{figures/5_Ascent/FirstStage}
\caption{}
\label{fig:FirstStage}
\end{figure}


\section{Flyback trajectories }

-first stage to subsonic flight? 
-second stage
Abort analysis?