% proci.tex

\cleardoublepage
\chapter{LODESTAR}\label{chapter:LODESTAR}	
This chapter covers the optimal control program LODESTAR (Launch Optimisation and Data Evaluation for Scramjet Trajectory Analysis Research), which has been used to simulate the optimal trajectories of the rocket-scramjet-rocket system. The dynamic model and structure of LODESTAR are presented, as well as the set-up of LODESTAR for the rocket-scramjet-rocket trajectory optimisation, and the verification methods used to determine if a solution has converged correctly.

LODESTAR has been developed to aid with the simulation and trajectory optimisation of space launch systems. 
LODESTAR is MATLAB based and utilises GPOPS-2[CITEXX], a proprietary pseudospectral method optimisation package.
 LODESTAR optimises a trajectory towards a user-defined objective function, such as maximum payload-to-orbit, subject to constraints which bound the operational region of a vehicle. Both rocket-powered and scramjet-powered vehicles are accurately modelled within LODESTAR in 6 degrees of freedom. LODESTAR contains multiple modes configured for the SPARTAN launch system, which are able to optimise trajectories for;

\begin{enumerate}
 \item The ascent of the first stage rocket.
 \item The ascent of the second stage scramjet-powered accelerator.
 \item The flyback of the second stage scramjet-powered accelerator.
 \item The ascent of the third stage rocket.
 \item Combined trajectories of multiple stages in any combination.
\end{enumerate}

Figure \ref{fig:FlowChartSmall} illustrates a simplified iteration of the pseudospectral solver. GPOPS-2 provides an initial guess of the solution to the external modules.
These modules calculate the vehicle aerodynamic and engine performance at each point along the trajectory, along with atmospheric conditions. This data is then used to calculate the dynamics of the vehicle along the trajectory. The imposed constraints and cost are then passed to IPOPT and evaluated, to compute the feasibility and optimality of the solution. Based on this evaluation, the trajectory guess is then updated to a new iteration, which is processed by GPOPS. This process repeats until the solver reaches a predefined tolerance of optimality.
\begin{figure}[ht]
	\centering
	\includegraphics[width=0.75\linewidth]{figures/4_LODESTAR/FlowChartSmall}
	\caption{The optimisation process.}
	\label{fig:FlowChartSmall}
\end{figure}
Due to the nature of the pseudospectral method, it is possible for GPOPS-2 to not be able to converge to a physically valid or optimal solution. 
LODESTAR contains a number of verification modules which assess the optimised trajectory solution to ensure that the solution is indeed an optimal trajectory, and that the dynamics of the solution are accurate. 



\section{Vehicle Simulation}


Each of the vehicles within the rocket-scramjet-rocket launch system are simulated by establishing a set of dynamic equations which fully describe the motion of the vehicle in terms of the time, states ($\mathbf{x}$), and controls ($\mathbf{u}$) of the system;
\begin{equation}
\dot{\textbf{x}}(t) = f[t,\textbf{x}(t),\textbf{u}(t)].
\end{equation}
 The states and controls are the variables which define the time dependent physical characteristics of the system. The state variables are dependent on the controls and the system dynamics, while the control variables are the variables which drive the behaviour of the system and are independently variable.  
 
These dynamic equations consist of the equations of motion of the vehicle, as well as other important time varying parameters, such as fuel mass flow rate. 
The dynamic equations are defined by the coordinate system, and the outputs of each vehicle model. These are nonlinear equations which depend on the interpolation of data sets which supply the atmospheric, aerodynamic and propulsion characteristics of each vehicle. 




\subsection{6DOF Equations of Motion}


The dynamics of the vehicle are calculated in six degrees of freedom, with yaw constrained to zero. 
The dynamics of all stages are calculated using an geodetic rotational reference frame, written in terms of the angle of attack $\alpha$, bank angle $\eta$, radius from centre of Earth $r$, longitude $\xi$, latitude $\phi$, flight path angle $\gamma$, velocity $v$ and heading angle $\zeta$. The equations of motion are \cite{Josselyn2002a}:
\begin{figure}[ht]
	\centering
	\includegraphics[width=0.7\linewidth]{figures/4_LODESTAR/global}
	\caption{The Earth-fixed components of the geodetic rotational coordinate system.}
	\label{fig:global}
\end{figure}
\begin{figure}[ht]
	\centering
	\includegraphics[width=0.9\linewidth]{figures/4_LODESTAR/Axes}
	\caption{The vehicle-based components of the coordinate system.}
	\label{fig:Axes}
\end{figure}


\begin{equation}
\dot{r} = v \sin \gamma
\end{equation}

\begin{equation}
\dot{\xi} = \frac{v\cos \gamma \cos \zeta}{r \cos \phi}
\end{equation}

\begin{equation}
\dot{\phi} = \frac{v\cos\gamma\sin\zeta}{r}
\end{equation}
\begin{equation}
\dot{\gamma} = \frac{T\sin\alpha \cos\eta}{mv} + (\frac{v}{r}-\frac{\mu_E}{r^2 v})\cos\gamma + \frac{L}{mv}
+ \cos\phi[2\omega_E \cos\zeta + \frac{\omega_E^2 r}{v}(\cos\phi\cos\gamma+\sin\phi\sin\gamma\sin\zeta)]
\end{equation}
\begin{equation}
\dot{v} = \frac{T\cos\alpha}{m}-\frac{\mu_E}{r^2}\sin\gamma - \frac{D}{m}
+ \omega_E^2 r\cos\phi(\cos\phi\sin\gamma-\sin\phi\cos\gamma\sin\zeta)
\end{equation}
\begin{equation}
\dot{\zeta} = \frac{T\sin\alpha \sin\eta}{mv \cos \gamma}-\frac{v}{r}\tan\phi\cos\gamma\cos\zeta +2\omega_E\cos\phi\tan\gamma\sin\zeta - \frac{\omega_E^2 r}{v\cos\gamma}\sin\phi\cos\phi\cos\zeta-2\omega_E\sin\phi 
\end{equation}






\section{Optimal Control Problem Structure}

The pseudospectral method used by GPOPS-2 is described in detail in section \ref{sec:Optimisation}. Practically, the implementation of optimal controls involves the specification of the dynamics of the system to be optimised, as well as the set of constraints and objectives which govern the optimisation problem. 
 Together, these define the optimisation problem being solved.

\noindent \textit{Cost Function}

\noindent The cost function, $J$, defines the target of the optimisation problem. 
This cost function may be any function which is defined by the states or controls of the optimisation problem. The cost function is defined as follows:
\begin{equation} \label{eq:cost}
J(t,\textbf{x}(t),\textbf{u}(t)) = M[t,\textbf{x}(t_f),\textbf{u}(t_f)] +   \int_{t_0}^{t_f} P[\textbf{x}(t),\textbf{u}(t)] dt, \quad t \in [t_0,t_f],
\end{equation}
where $M$ is the terminal cost function and $P$ is the time integrated cost. 

\noindent \textit{Dynamic Constraints}

\noindent The constraints impose various conditions on the optimisation problem.
The optimisation problem is subject to a set of dynamic constraints, which describe the behaviour of the system over the solution space:
\begin{equation} \label{eq:state}
\dot{\textbf{x}}(t) - f[t,\textbf{x}(t),\textbf{u}(t)] = 0.
\end{equation}
Implementing the dynamics as constraints allows each state variable to be approximated separately, and gives the optimiser some freedom to explore each state variable independently, greatly increasing the robustness of the optimal control problem. However, implementing the dynamics as constraints means that the dynamics do not have to be physically valid at all times during the process of searching for an optimal solution. A violation of the physical dynamics constraints can cause potential complications for the computational model of the vehicle. Much of the design of the vehicle model in this study is driven by the need for smooth, continuous interpolation schemes, which cover the entire possible operational range of the vehicle, ie. even if the solution is well within the range of all input data sets, the solver must be able to explore all regions within the set bounds. 

\noindent \textit{Bounds and Path Constraints}

\noindent Inequality constraints define the bounds of each state, as well as any path constraints.
The bounds directly confine the state and control variables to prescribed values. This serves the purpose of limiting the search space to the physically possible (eg. constraining altitude to be greater than ground level), constraining the vehicle within its performance limits (eg. limiting the angle of attack), and improving computational efficiency by ensuring that the optimiser is constrained to a reasonable solution space:
\begin{eqnarray}
\mathbf{b}_{min} \leq \textbf{x}(t),\textbf{u}(t) \leq \mathbf{b}_{max}.
\end{eqnarray}
The path constraints are inequality constraints which consist of functions based on the states and controls of the system. Path constraints are generally used to impose physical limitations on the system such as structural, aerothermodynamic or pathing limitations:
\begin{eqnarray}
\mathbf{\lambda}[t,\textbf{x}(t),\textbf{u}(t)] \leq \textbf{0}.
\end{eqnarray}

\noindent \textit{Event Constraints}

\noindent The event constraints define the states at the start and end points of a trajectory or phase:
\begin{equation}
\mathbf{\psi}_0[\textbf{x}(t_{0}), t_{0}] = \textbf{0},
\end{equation}
\begin{equation} \label{eq:2}
\mathbf{\psi}_f[\textbf{x}(t_{f}), t_{f}] = \textbf{0}.
\end{equation}
These constraints determine the initial and terminal conditions of the optimisation problem. Additionally, if the problem has multiple phases, these constraints are used to couple the states and time of each phase to the preceding and following phases as follows:
\begin{equation}
\textbf{x}_{f,1} - \textbf{x}_{0,2} = 0,
\end{equation}
\begin{equation}
\textbf{t}_{f,1} - \textbf{t}_{0,2} = 0.
\end{equation}
















\subsection{Trajectory Connection Points}
The optimisation of a large, multi-vehicle launch trajectory requires that the optimal control problem be broken down into multiple segments. This segmentation is performed in order to assist the convergence of the optimal control solver, by ensuring that the dynamics of the underlying model are as smooth and continuous as possible across each segment. 
For a launch system, discontinuities in the system dynamics generally arise when the aerodynamics, mass and propulsion mode of a launch vehicle change significantly between stages or flight modes. 
If a vehicle model with large discontinuities is implemented directly into a single phase application of the pseudospectral method, it is likely to cause significant convergence issues, as the system dynamics will be unable to be approximated by the underlying polynomial of the pseudospectral method. 
 \begin{figure}[ht]
 	\centering
 	\includegraphics[width=1.\linewidth]{figures/4_LODESTAR/Traj}
 	\caption{Illustration of the segmented launch profile.}
 	\label{fig:Traj}
 \end{figure}
 
 To allow the trajectory profile to be formulated as an optimal control problem, the trajectory of the rocket-scramjet-rocket launch system has been broken down into the seven segments shown in Figure \ref{fig:Traj}. 
  The segments have been separated into two groups; controlled segments which form the phases of the optimal control problem, and segments without control which are simulated separately, to increase computational efficiency and improve the convergence of the optimal control solver.  
 Segments \textcolor{red}{\rom{2}-\rom{5}} are controlled by various combinations of angle of attack, bank angle and throttle, and are implemented as the phases of the optimisation problem. These phases are; The 1st stage pitching ascent; the 2nd stage ascent; the 2nd stage return flight; and the 3rd stage powered ascent.
 Segments \textcolor{red}{\rom{1}},\textcolor{red}{\rom{6}} and \textcolor{red}{\rom{7}} are segments without direct control, which are simulated using forward time stepping methods. 
 These phases are; the pre-pitch segment of the first stage; the unpowered section of the third stage ascent; and the final Hohmann transfer to orbit. 
 Each segment is connected through a set of conditions, which ensure that the trajectory of the vehicle is continuous, and that the trajectory that is being simulated is the one that is intended. 
  The optimal control problem phases are connected through the use of initial and end discontinuity constraints on each phase to be coupled, ie $\textbf{x}_{f1} = \textbf{x}_{02}, t_{f1} = t_{02}$, while the forward simulated segments are simply initiated and terminated at set conditions. 
 The segment coupling conditions are described in Table \ref{tab:constraints}.






\begin{table}[H]


\begin{tabularx}{\linewidth}{|X|X|X|}
	\hline \textbf{Section} & Initial Conditions & End Conditions  \\ 
	\hline $1^{st}$ Stage Vertical Ascent (\textcolor{red}{\rom{1}}) & Must start at rest, at the predefined launch site. & Fly until pitchover conditions are met. \\ 
	\hline $1^{st}$ Stage Pitching Ascent (\textcolor{red}{\rom{2}}) & Start at pitchover conditions & - \\ 
	\hline $2^{nd}$ Stage Ascent (\textcolor{red}{\rom{3}}) & Must begin at $1^{st}$ stage pitching ascent end conditions. & - \\ 
	\hline $2^{nd}$ Stage Return (\textcolor{red}{\rom{4}}) & Must begin at $2^{nd}$ stage ascent end conditions. & Must approach landing conditions at the initial launch site. \\ 
	\hline $3^{rd}$ Stage Powered Ascent (\textcolor{red}{\rom{5}}) & Must begin at $2^{nd}$ stage ascent end conditions.  & Must produce exoatmospheric flight at the termination of stage \rom{6}.  \\ 
	\hline $3^{rd}$ Stage Unpowered Ascent (\textcolor{red}{\rom{6}}) & Must begin at $3^{nd}$ stage powered ascent end conditions.  & Terminates when flight is parallel with Earth's surface.  \\ 
	\hline $3^{rd}$ Stage Hohmann Transfer (\textcolor{red}{\rom{7}}) & Must begin at $3^{rd}$ stage unpowered ascent end conditions. & Must attain prescribed orbit.  \\ 
	\hline 
	
\end{tabularx} 
\caption{Segment coupling conditions for combined trajectory optimisation.}
\label{tab:constraints}

\end{table}



\subsubsection{\textcolor{red}{\rom{1}.} First Stage Vertical Ascent}

LODESTAR optimises the ascent of the first stage rocket in two sections; pre and post-pitchover.
Initially, the first stage rocket is launched vertically, and continues vertically for a small amount of time, until pitchover is initiated. 
During the vertical launch the rocket is assumed to need no control, and is held at 0$^\circ$ angle of attack. 
The pitchover is defined to occur at 90m altitude and 30m/s velocity, rather than at a specific time.
This allows the pre-pitchover phase to be simulated separately, after the optimisation has been completed. 
If the initial mass is variable, the pre-pitchover trajectory is calculated to match the pitchover mass, and the initial launch altitude will necessarily be slightly variable. This variation in launch altitude is small, on the order of ten meters or less. 

\subsubsection{\textcolor{red}{\rom{2}.} First Stage Pitching Ascent}


The post-pitchover trajectory is an angle of attack controlled phase in the optimisation routine, which is simulated from pitchover until second stage separation. During this phase, the launch system is allowed to fly at negative angles of attack, to assist in pitching. 
The fuel mass of the first stage rocket is unconstrained.
This approach is used on account of the selected first stage being able to reach the required range of altitudes and flight angles with only small fuel mass variations. 
Nevertheless, small variation in fuel mass can have an important effect on the capabilities of the first stage, influencing the velocity achievable at first to second stage separation, as well as the rate at which the rocket is able to pitch, and consequentially, the altitude and flight path angle range of the first stage.
It is useful in the preliminary design stages to be able to vary the mass of the first stage vehicle during the optimisation process, allowing a less trial-and-error approach than selecting a starting fuel mass manually.



\begin{table}[H]
\centering
\begin{tabular}{|c|c|c|}
	\hline Optimisation Parameter  & Associated Variables & Allowable Values\\
	\hline Initial Constraints  & Velocity & 30m/s\\ & Altitude& 90m \\ & Latitude & $-12.16^\circ$ \\& Longitude & 136.75$^\circ$\\ & Trajectory Angle & 89.9$^\circ$\\ & Angle of Attack& 0$^\circ$\\
	\hline Terminal Constraints & $\textbf{x}_{f\textrm{\rom{2}}} - \textbf{x}_{0\textrm{\rom{3}}}$ & 0\\ & $t_{f\textrm{\rom{2}}} - t_{0\textrm{\rom{3}}}$ & 0\\
	\hline Path Constraints & Dynamic Pressure & 0kPa - 50kPa\\ 
		\hline Control Variables & $\ddot{\alpha}$ &\\ 
		\hline State Variables & Altitude & \\ & Velocity& \\  & Latitude& \\  & Longitude& \\  & Trajectory Angle& \\  & Heading angle& \\  & Total mass& \\  & Angle of Attack ($\alpha$)&  $-5^\circ$ - 5$^\circ$\\  & $\dot{\alpha}$& \\ 
	\hline 
\end{tabular} 

\caption{Optimisation setup of the first stage phase. \textcolor{red}{The other tables in this section will be modified to conform to this one. But I am currently unsure whether this table layout is clear.}}

\end{table}



\subsection{\textcolor{red}{\rom{3}.} Second Stage Ascent Trajectory}

The ascent trajectory of the SPARTAN is controlled using angle of attack, and bank angle. This trajectory is constrained to a maximum dynamic pressure of 50kPa, corresponding to the maximum structural limits of the vehicle. During the ascent, the engines are assumed to be operating at an equivalence ratio of 1 at all times. 

\begin{table}[H]
\begin{tabular}{|c|c|}
	\hline Initial Constraints  & Velocity \\ & Fuel Mass  \\ & Latitude \\ & Longitude \\ 
	\hline Terminal Constraints & Fuel mass \\ & Heading Angle \\ 
	\hline Path Constraints & Dynamic Pressure \\ 
	\hline Target Cost & Maximum Payload-to-Orbit \\ 
			\hline Control Variables &  \\ 
			\hline State Variables &  \\ 
	\hline 
\end{tabular} 

\end{table}

\subsection{\textcolor{red}{\rom{4}.} Second Stage Return Trajectory}
After releasing the third stage rocket, the scramjet-powered second stage must return back to the initial launch site.
During the fly-back, the SPARTAN cannot exceed its dynamic pressure limit of 50kPa. 
 The end state is constrained to a minimum of $-20^\circ$ trajectory angle, which is assumed to be an appropriate trajectory angle for approach to a landing strip. All other state variables are left unconstrained at the end point. It is assumed that for an optimal trajectory, the SPARTAN will end its return at the minimum energy state possible, corresponding to the minimum velocity and altitude attainable while satisfying the end latitude and longitude constraints. 
 
During the return, the C-REST engines are able to be throttled on and off. The throttle is set as a control variable, variable between 0 and 1, where 1 represents the maximum equivalence ratio at that point. The fuel mass flow rate is scaled linearly with the throttle:  
\begin{equation}
\dot{m}_{fuel} = \dot{m}_{fuel,max}throttle,
\end{equation}
and the thrust of the engine is assumed to scale linearly with the fuel mass flow rate. 

\begin{table}[H]
\begin{tabular}{|c|c|}
	\hline Initial Constraints  & Altitude \\ & Velocity\\ & Flight Path Angle\\ & Heading Angle\\ & Latitude\\ & Longitude\\ 
	\hline Terminal Constraints &  Distance From Launch Site \\ 
	\hline Path Constraints & Dynamic Pressure \\ 
	\hline Target Cost & Minimum End Velocity \\ 
				\hline Control Variables &  \\ 
				\hline State Variables &  \\ 
	\hline 
\end{tabular} 
\end{table}



\subsection{\textcolor{red}{\rom{5}.} Third Stage Powered Ascent}

The trajectory of the third stage rocket is only directly optimised during the powered section of its trajectory. During powered flight, the engine is used to control the third stage vehicle. After this point, the engine is cut, and the third stage is assumed to not have sufficient aerodynamic control to manoeuvre. 

The third stage rocket is constrained to an angle of attack of less than 20$^\circ$. This is assumed to be the maximum controllable angle of attack possible for the third stage rocket.   
Additionally, a maximum normal force restriction is placed on the third stage. Previous studies flew the third stage rocket at a constant 10$^\circ$ angle of attack, and initially released the rocket at 50kPa[CITEXX]. 
For this study, it is assumed that 10$^\circ$ angle of attack produces the maximum allowable normal force at 50kPa, to prevent the rocket from being released into an environment which could exceed its structural limitations. The maximum allowable normal force is calculated at the release Mach number, and set as a path constraint. 
\begin{table}[H]
\begin{tabular}{|c|c|}
	\hline Initial Constraints  & \\ 
	\hline Terminal Constraints & None \\ 
	\hline Path Constraints & Angle of Attack \\ 
	\hline Target Cost &  \\ 
				\hline Control Variables &  \\ 
				\hline State Variables &  \\ 
	\hline 
\end{tabular} 
\end{table}

\subsection{\textcolor{red}{\rom{6}.} Third Stage Unpowered Ascent}

The unpowered section of the trajectory is simulated from the end of the controlled section of the trajectory, using a second order Taylor series approximation. This integration ceases when the flight path angle reaches 0$^{\circ}$.
During this phase, the heat shield is released once the rocket has reached a dynamic pressure of 10Pa, where it is assumed that atmospheric effects will have ceased to have a major thermal effect.  As the third stage is required to deliver the payload into heliosynchronous orbit, the third stage must achieve an inclination of 97.63$^\circ$ at the end of this phase.



\subsection{\textcolor{red}{\rom{7}.} Hohmann Transfer}
After the rocket has attained exoatmospheric flight parallel to the Earth's surface, a circularisation burn is performed. This circularisation burn takes the third stage rocket into low orbit around the Earth. 
However, in order to reach a heliosynchronous orbit of 567km, the orbit of the third stage rocket must be raised. 
To this end, the final manoeuvre performed by the third stage rocket is a Hohmann transfer. A Hohmann transfer is the most fuel efficient way to raise a spacecraft from one circular orbit to another. 
Following circularisation, the third stage engine is reignited (or remains ignited) and the third stage manoeuvres into an appropriate elliptical orbit. 

\begin{figure}
\centering
\includegraphics[width=0.7\linewidth]{figures/4_LODESTAR/Hohmann}
\caption{The Hohmann transfer manoeuvre.}
\label{fig:Hohmann}
\end{figure}




Circularisation burn
\begin{equation}
\Delta V_{12} = \sqrt{\dfrac{\mu}{r_2}} - V_1
\end{equation}

begin hohmann transfer
\begin{equation}
\Delta V_{23} = \sqrt{\dfrac{\mu}{r_2}} \left( \sqrt{\dfrac{2r_4}{r_2 + r_4}} -1 \right)
\end{equation}

insertion burn
\begin{equation}
\Delta V_{34} = \sqrt{\dfrac{\mu}{r_4}} \left(1- \sqrt{\dfrac{2r_2}{r_2 + r_4}}  \right)
\end{equation}


The mass of the third stage rocket at each burn is calculated using the Tsiolkovsky rocket equation:

\begin{equation}
m_2 = \frac{m_{1f}}{\exp^{\frac{V_{12}}{I_{SP} \cdot g_0}}}
\end{equation}
\begin{equation}
m_3 = \frac{m_{2}}{\exp^{\frac{V_{23}}{I_{SP} \cdot g_0}}}
\end{equation}
\begin{equation}
m_4 = \frac{m_{3}}{\exp^{\frac{V_{34}}{I_{SP} \cdot g_0}}}
\end{equation}
Finally, the payload-to-orbit is determined by removing the structural mass from the total mass of the vehicle at the end of the Hohmann transfer. The remaining mass is taken to be the payload-to-orbit capability of the vehicle.
\begin{equation}
m_{payload} = m_4 - m_{struct}
\end{equation}







\section{The Optimisation Process}

section on how the modules are brough together and evaluated by gpops and ipopt


Figure \ref{fig:AscentFlowchart} shows the information flow within LODESTAR. 

\begin{landscape}% Landscape page
	\begin{figure}[ht]
		\centering
		\includegraphics[width=0.9\linewidth]{"figures/4_LODESTAR/Ascent Flowchart"}
		\caption{The process of the rocket-scramjet-rocket trajectory optimisation.}
		\label{fig:AscentFlowchart}
	\end{figure} 
\end{landscape}



\section{Optimality Conditions}

LODESTAR provides the capacity to verify the optimal solution provided by the pseudospectral method solver. This partial verification is used to determine whether the pseudospectral method solver has converged close to an optimal solution of the nonlinear programming problem. It is particularly useful to verify that the optimality and constraint tolerances which have been chosen are sufficiently small, or to check whether the pseudospectral method solver has approached an optimal solution in the case that the defined tolerances are not able to be reached.   
This partial verification is achieved through the examination of four key metrics; the IPOPT constraint violation and dual infeasibility parameters; the Hamiltonian necessary condition for optimality; the state derivatives; and finally a forward simulation. 

The first metrics to be checked are the IPOPT constraint violation ($inf\textrm{-}pr$) and dual infeasibility parameter ($inf\textrm{-}du$)\cite{Kawajir2010}. The constraint violation parameter is a measure of the infinity-norm ($L_\infty\textrm{-}norm$) of the constraints of the problem\cite{Kawajir2010}. This factor must be suitably small in order to indicate that the constraints of the problem to have been met. While the permissible magnitude of this factor changes with each individual problem, it is always desirable for this factor to be as small as possible. For this study only solutions with $L_\infty\textrm{-}norm(inf\textrm{-}pr) < 10^{-4} $ are accepted as having satisfied the imposed constraints. The dual infeasibility is a Karush-Kuhn-Tucker condition\cite{Hindi2006}, which is a necessary condition for optimality. A dual feasible solution indicates that the dual problem is at least a lower bound on the optimal solution, $p^\star$, ie. $p^\star \geq g(\lambda,v)$. For more details on duality see \cite{Hindi2006}.
A low dual infeasibility indicates that the solution has approached an optimal solution. Again, the magnitude of this value is variable with each problem, though as a problem becomes more complex, the ability to converge towards an optimal solution diminishes. For the problem in this study, $L_\infty\textrm{-}norm(inf\textrm{-}du) < 0 $ are accepted due to the highly complex nature of the vehicle model. In this study it is accepted that a given solution may not approach the global optimum, and multiple solutions are calculated to mitigate the error caused by the problem complexity, with the 'most optimal' solution selected. 


The Hamiltonian of the optimal control problem is defined as 
\begin{equation}
H(x(t),u(t),\lambda(t),t) = \lambda^T(t)f(x(t),u(t)) + L(x(t),u(t)).
\end{equation}
The Hamiltonian of the optimal control problem is investigated as a partial verification that the first order necessary conditions hold. Due to the unconstrained end time of the trajectory problems, $H\equiv 0 $ \cite{Pucci2007}. 
This is calculated using LODESTAR and the Hamiltonian condition is able to be verified. The Hamiltonian will likely not be exactly equal to zero along the trajectory. This is due to the heuristic nature of the solver, which will approach close to an optimal solution, but never reach it exactly. A sufficiently small Hamiltonian indicates that the end solution approaches an optimal solution, and may be a candidate as an optimised trajectory case. 


The pseudospectral method considers the dynamics of the system as constraints on the optimal control problem, and solves across the entire trajectory simultaneously. This causes the physical system dynamics to have an associated margin of error, ie. $\dot{x} = f(x)$ will only hold to a certain degree of accuracy. For a well converged solution, this margin of error will be negligibly small, and the dynamics of the system will be consistent with realistic Newtonian dynamics. However, when the problem is not well converged, the dynamics of the system may have a large error.
A check is performed on each state to affirm that the derivative of the approximated state is equal to the derivative supplied by the vehicle model. This checks that the solver has converged to a solution which satisfies the vehicle dynamics at each individual node. 
The state feasibility of the solution is checked through a comparison of the state derivatives, $\dot{x} = f(x,u)$. $\dot{x}$ is first determined through numerical differentiation of the state variables over the solution time. Then $f(x,u)$ is determined using the dynamics of the system and vehicle model, in the same way that $f(x,u)$ is input to the pseudospectral solver. Examination of the error between the 'expected' state derivatives, and the numerical approximation of the derivatives, $\dot{x} - f(x,u)$, allows the accuracy of the system dynamics to be verified. 



 The final verification check is a full forward simulation. This forward simulation starts at the initial conditions prescribed by the pseudospectral method solver, and propagates the dynamics of the system forward in time using numerical approximation. The forward simulation uses the optimised control variables as the only input. 
This checks that the flight path will follow the optimised path using the calculated control inputs. This is the most complete test of the optimal solution. However, in some cases calculating a forward solution may be problematic. The pseudospectral method has a limited number of nodes, potentially spread across relatively large time steps. Due to the high accuracy of the polynomial approximation, the pseudospectral method is able to maintain accuracy over large time steps. However, a forward simulation necessarily has less accuracy than the spectral method, and may interpolate differently when applied to the optimal solution, causing minor deviations. This particularly occurs when the states or controls are changing rapidly. These minor deviations may propagate themselves forward, causing significant deviation in the forward simulation. In some cases, this can also cause the forward simulation to appear close to the optimial solution, when it should deviate. For this reason, a forward simulation is a good final check of an optimal solution, but knowledge of the system dynamics must be used, along with the other verification methods, to ascertain if a solution has converged sufficiently. 

\textcolor{red}{It may be useful to include example images here}

