% proci.tex

\cleardoublepage
\chapter{LODESTAR}\label{chapter:experimental-results}

 \textcolor{red}{Ingo said maybe have an example in here illustrating the optimisation. maybe something from a textbook. }	

The program LODESTAR (Launch Optimisation and Data Evaluation for Scramjet Trajectory Analysis Research) has been developed to aid with the simulation and trajectory optimisation of space launch systems. LODESTAR is a MATLAB based trajectory optimiser which utilises DIDO, a proprietary pseudospectral method optimisation package as well as MATLAB's inbuilt SQP solver. LODESTAR optimises a trajectory towards a user-defined objective function, such as constant dynamic pressure or maximum payload-to-orbit.  LODESTAR accurately models both rocket-powered and scramjet-powered vehicles in 5 degrees of freedom. LODESTAR contains multiple modules configured for the SPARTAN launch system, which are able to optimise trajectories for;
\begin{enumerate}
 \item The ascent of the first stage rocket.
 \item The ascent of the second stage scramjet-powered accelerator.
 \item The flyback of the second stage scramjet-powered accelerator.
 \item The ascent of the third stage rocket.
\end{enumerate}

\subsection{Optimal Control}
The pseudospectral method and direct single shooting techniques used by LODESTAR are described in detail in sections X-X. Practically, the implementation of these techniques involves the specification of the set of constraints and objectives which govern the optimisation problem. These constraints inform the optimiser of the limits of the optimisation, and perform the functions of bounding the logical search space (eg. constraining altitude to be greater than ground level) as well as constraining the vehicle within its performance limits (eg. limiting the angle of attack). These constraints also come in the form of initial or terminal constraints, which define the initial conditions of the trajectory as well as any conditions which the trajectory must meet at termination. 

The pseudospectral method requires the specification of a set of 'primal variables'. These primal variables describe the physical dynamics of the system. In the pseudospectral method, the dynamics of the system are used as constraints on the optimal control problem;
\begin{equation} \label{eq:state2}
\dot{\textbf{x}}(\tau) = f[\textbf{x}(\tau),\textbf{u}(\tau)].
\end{equation}
Implementing the dynamics as constraints allows the optimiser to explore each primal variable independently, greatly increasing the robustness of the optimal control problem. However, the constraints may be violated by the optimiser in the process of searching for an optimal solution. A violation of the physical dynamics constraints means that the dynamics of the system may not hold throughout the solution process, causing potential complications for the computational model of the vehicle. Much of the design of the vehicle simulation in this study is driven by the need for smooth, continuous interpolation schemes, and viable extrapolation regions ie. even if the solution is well within the range of all input data sets, the solver must be able to explore all regions within the set bounds. 

\subsection{Dynamic Model}
The drag and lift produced by each stage of the vehicle are calculated using the standard definition of the aerodynamic coefficients:

\begin{equation}
F_d = \frac{1}{2}\rho c_d v^2 A ,
\end{equation}
\begin{equation}
F_L = \frac{1}{2}\rho c_L v^2 A .
\end{equation}

The dynamics of all stages are calculated using an geodetic rotational reference frame, written in terms of the radius from centre of Earth $r$, longitude $\xi$, latitude $\phi$, flight path angle $\gamma$, velocity $v$ and heading angle $\zeta$. The equations of motion are \cite{Josselyn2002a}:


\begin{equation}
\dot{r} = v \sin \gamma
\end{equation}

\begin{equation}
\dot{\xi} = \frac{v\cos \gamma \cos \zeta}{r \cos \phi}
\end{equation}

\begin{equation}
\dot{\phi} = \frac{v\cos\gamma\sin\zeta}{r}
\end{equation}
\begin{equation}
\dot{\gamma} = \frac{T\sin\alpha \cos\eta}{mv} + (\frac{v}{r}-\frac{\mu_E}{r^2 v})\cos\gamma + \frac{L}{mv}
+ \cos\phi[2\omega_E \cos\zeta + \frac{\omega_E^2 r}{v}(\cos\phi\cos\gamma+\sin\phi\sin\gamma\sin\zeta)]
\end{equation}
\begin{equation}
\dot{v} = \frac{T\cos\alpha}{m}-\frac{\mu_E}{r^2}\sin\gamma - \frac{D}{m}
+ \omega_E^2 r\cos\phi(\cos\phi\sin\gamma-\sin\phi\cos\gamma\sin\zeta)
\end{equation}
\begin{equation}
\dot{\zeta} = \frac{T\sin\alpha \sin\eta}{mv}-\frac{v}{r}\tan\phi\cos\gamma\cos\zeta +2\omega_E\cos\phi\tan\gamma\sin\zeta - \frac{\omega_E^2 r}{v\cos\gamma}\sin\phi\cos\phi\cos\zeta-2\omega_E\sin\phi 
\end{equation}



	Flow chart of modules
	details of simulation (5DOF geodetic rotational)
	details of limits
	
	verification methods
	-hamiltonian/costates
	-complementary conditions
	-forward sim (for sanity checking, will need to detail deficiencies in this)
	-forward integration
	-logic check (ie solver is still a heuristic process, run multiple times with varying guess. Is solution logical?)

-Geodetic rotational coordinates 
-put coordinate system here
-Pontani has a typo is his paper remember
-add in portion of lift going towards changing heading angle due to roll. This is just simple centripetal force. 
$F=mr\omega^2$ $v=\omega r$



- outline the pseudospectral method, as implemented in GPOPS? ie with stage definitions

\subsection{Trajectory Connection Points}
image here detailing the trajectory and separation points, including what the constraints of each trajectory are. Maybe number these separation points, and refer to these numbers in th constraint table for each trajectory stage. 

Flow chart here of how the main LODESTAR modules connect with reference to the trajectory diagram as well

\begin{figure}
\centering
\includegraphics[width=1.\linewidth]{figures/4_LODESTAR/Traj}
\caption{}
\label{fig:Traj}
\end{figure}


\subsection{First Stage Trajectory}
-detail the vehicle dynamics with a free body diagram 


LODESTAR is able to optimise the first stage of a launch vehicle, for an angle of attack controlled trajectory, from launch to a pre-defined end point. 
LODESTAR is able to optimise for either a maximum velocity or minimum mass optimisation objective. 

A maximum velocity case is desired when a specific first stage vehicle design is being investigated. 
 
A minimum mass objective is applicable when the first stage trajectory has a pre-defined end goal. This is the case with the SPARTAN vehicle where the SPARTAN scramjet accelerator is to be released at its minimum operating conditions at close to horizontal flight. 
A variable mass for the first stage launch vehicle is desired as the mass has large effects on the dynamics of th vehicle, effecting the trajectory angle change rate, as well as the acceleration and time of flight of the vehicle.
It is useful in the preliminary design stages to be able to optimise the mass of the first stage vehicle, allowing a less trial-and-error approach.
Note that in the minimum mass case, the launch altitude is slightly variable, as LODESTAR starts optimisation from the pitchover time, and the pre-pitchover trajectory is calculated to match the pitchover mass. 

\begin{tabular}{|c|c|}
	\hline Input  & Contains\\ 
	\hline Aerodynamic Database  & \\ 
	\hline 
\end{tabular} 


The first stage is launched from an area in northern Queensland.

\begin{tabular}{|c|c|}
	\hline Initial Constraints  & \\ 
	\hline Terminal Constraints &  \\ 
	\hline Path Constraints &  \\ 
	\hline Target Cost &  \\ 
	\hline 
\end{tabular} 

\subsubsection{Control Variables}

\subsubsection{Primal Variables}

\subsection{Second Stage Trajectory}

-detail the vehicle dynamics with a free body diagram 

LODESTAR is able to optimise the trajectory of airbreathing accelerators in the supersonic or hypersonic regime. LODESTAR is able to optimise for a range of optimisation metrics, including a maximum payload-to-orbit and constant dynamic pressure.  
The trajectory of the SPARTAN scramjet-powered accelerator has been simulated in LODESTAR. 

\begin{tabular}{|c|c|}
	\hline Initial Constraints  & Velocity \\ & Fuel Mass  \\ & Latitude \\ & Longitude \\ 
	\hline Terminal Constraints & Fuel mass \\ & Heading Angle \\ 
	\hline Path Constraints & Dynamic Pressure \\ 
	\hline Target Cost & Maximum Payload-to-Orbit \\ 
	\hline 
\end{tabular} 

\subsubsection{Control Variables}

\subsubsection{Primal Variables}
primal variable limits:

\subsection{Second Stage Return Trajectory}
After releasing the third stage rocket, the scramjet-powered second stage must return back to an area close to the initial launch site.
During the flyback, the SPARTAN cannot exceed its dynamic pressure limit of 50kPa. 
The SPARTAN must land on the ground with minimum velocity, within close proximity to the launch area. It is assumed that a landing strip is available at the spot where the SPARTAN lands. The SPARTAN is required to land within a set radius of the launch site. This constraint is;
\begin{equation}
(\phi_{end} - \phi_{launch})^2 + (\xi_{end} - \xi_{launch})^2 - r^2 \leq 0
\end{equation}



Summary Table

\begin{tabular}{|c|c|}
	\hline Initial Constraints  & Altitude \\ & Velocity\\ & Flight Path Angle\\ & Heading Angle\\ & Latitude\\ & Longitude\\ 
	\hline Terminal Constraints &  Distance From Launch Site \\ 
	\hline Path Constraints & Dynamic Pressure \\ 
	\hline Target Cost & Minimum End Velocity \\ 
	\hline 
\end{tabular} 

\subsubsection{Control Variables}


\subsection{Third Stage Trajectory}

-detail the vehicle dynamics with a free body diagram 

\begin{tabular}{|c|c|}
	\hline Input  & Contains\\ 
	\hline Aerodynamic Database  & Mach Number, AoA, CA, CN, CD, CL, cP\\ 
	\hline 
\end{tabular} 

\begin{tabular}{|c|c|}
	\hline Output  & Contains\\ 
	\hline   & \\ 
	\hline 
\end{tabular} 

The third stage is required to deliver the payload into heliosynchronous orbit. The heliosynchronous orbit chosen is 566.89km. 

\begin{tabular}{|c|c|}
	\hline Initial Constraints  & \\ 
	\hline Terminal Constraints &  \\ 
	\hline Path Constraints & Angle of Attack \\ 
	\hline Target Cost &  \\ 
	\hline 
\end{tabular} 


-Detail the hohmann transfer
\subsubsection{Control Variables}

\subsection{Combined Second Stage Ascent \& Return  Trajectory}

\subsection{Validation}
\subsubsection{Optimality Conditions}
Hamiltonian = 0 because dH/dt = 0 (Hamiltonian does not depend explicitly on time) except at end due to final time being free.
Costates
Complementary conditions
\subsubsection{Forward Simulations}
-both the control check and derivative check
