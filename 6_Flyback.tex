% numerical.tex

\cleardoublepage
\chapter{Optimised Trajectory Including Fly-Back}\label{chapter:Flyback}
\textcolor{red}{exergy can be affected by the amount of fuel (as its variable), and keep this is mind during discussion. -i dont think using specific exergy is  useful though}


This chapter presents the maximum payload-to-orbit optimised trajectory of the rocket-scramjet-rocket launch system, including the fly-back of the SPARTAN. 
 Returning the SPARTAN for landing at the initial launch site allows for rapid refurbishment and re-use, and is one of the primary enabling factors in the cost efficient operation of the launch system. 
The fly-back of the SPARTAN requires a full turn-around of the SPARTAN, a glide which covers the necessary ground distance for return, and a deceleration which reduces the speed of the SPARTAN to landing approach velocity, while maintaining suitable high trajectory angle to allow for a controlled approach. 
The return of the SPARTAN to the initial launch site is included in the optimisation process, to maximise the overall efficiency of the launch system. 
This chapter includes a sensitivity analysis, with the same design parameters varied as in the preceding chapter. This sensitivity analysis allows the influence of the fly-back of the SPARTAN on the design sensitivities of the launch system to be assessed.


\section{Combined SPARTAN Ascent-Descent \& Third Stage}
\begin{table}[ht]
	\centering
\begin{tabular}{l c } 
	\hline \textbf{Trajectory Condition}
	&Standard

	\\
	\hline \textbf{Payload to Orbit (kg)}
	& \textbf{\PayloadToOrbitStandard}
	\\
	\textbf{Total $\eta_{exergy}$ (\%)}
	& \textbf{\totalExergyEffStandard}
	\\
	\hline 
	\textbf{1$^{st}$ Stage $\eta_{exergy}$ (\%)}
	& \textbf{\firstExergyEffStandard}
	\\
	\textbf{Separation Alt, 1$\rightarrow$2 (km)}
	& \firstsecondSeparationAltStandard
	\\
	\textbf{Separation v, 1$\rightarrow$2 (m/s)}
	& \firstsecondSeparationvStandard
	\\
	\textbf{Separation $\gamma$, 1$\rightarrow$2 (m/s)}
	& \firstsecondSeparationgammaStandard
	\\
	\hline 
	\textbf{2$^{nd}$ Stage $\eta_{exergy}$ (\%)}
	& \textbf{\secondExergyEffStandard}
	\\
	\textbf{Separation Alt, 2$\rightarrow$3 (km)}
	& \secondthirdSeparationAltStandard
	\\
	\textbf{Separation $v$, 2$\rightarrow$3 (m/s)}
	& \secondthirdSeparationvStandard
	\\
	\textbf{Separation $\gamma$, 2$\rightarrow$3 (deg)}
	& \secondthirdSeparationgammaStandard
	\\
	\textbf{Separation $q$, 2$\rightarrow$3(kPa)}
	& \secondthirdSeparationqStandard
	\\
	\textbf{2$^{nd}$ Stage L/D, 2$\rightarrow$3}
	& \secondthirdSeparationLDStandard
	\\
	\textbf{2$^{nd}$ Stage Flight Time (s)}
	& \secondFlightTimeStandard
	\\
	\hline 
	\textbf{3$^{rd}$ Stage $\eta_{exergy}$ (\%)}
	& \textbf{\thirddExergyEffStandard}
	\\
	\textbf{2$^{nd}$ Stage Return Fuel (kg)}
	& \returnFuelStandard
	\\
	\textbf{3$^{rd}$ Stage $t$, $q >$ 5kpa (s)}
	& \thirdqOverFiveStandard
	\\
	\textbf{3$^{rd}$ Stage max $\alpha$ (deg)}
	& \thirdmaxAoAStandard
	\\
	\textbf{3$^{rd}$ Stage Fuel Mass (kg)}
	& \thirdmFuelStandard
	\\
	\hline 
\end{tabular} 
\end{table}

The trajectory of the rocket-scramjet-rocket launch system has been optimised in LODESTAR, including the return of the SPARTAN to its initial launch site. The optimised trajectory is shown in Figure \ref{fig:GroundTrackStandard}. 
The rocket-scramjet-rocket launch system is shown to be able to launch a small satellite 
while flying-back the SPARTAN to the initial launch site location, and approaching the landing site at appropriately low altitude and velocity to allow for landing on a traditional runway. 
The optimised trajectory attains a payload mass to SSO of \PayloadToOrbitStandard kg. 
This indicates that the launch system utilising the SPARTAN is capable of successfully completing a small satellite launch mission which allows for rapid reusability of the SPARTAN. 
\begin{figure}[ht!]
	\centering
	\includegraphics[width=1\linewidth]{../LODESTAR_FINAL/Results/mode11/GroundTrackStandard}
	\caption{}
	\label{fig:GroundTrackStandard}
\end{figure}
The maximum payload-to-orbit is reduced by XX\% compared to the optimised trajectory result without fly-back. The benefits of flying back the SPARTAN to its initial launch site, compared to the alternative of transporting the SPARTAN back to the launch site from a remote landing, are likely to far outweigh this associated reduction in payload. 



\section{Ascent Trajectory}

 The initial launch for a maximum payload-to-orbit trajectory with SPARTAN fly-back is to the east.
 The first stage trajectory is very similar to that of the first stage launching the SPARTAN with no return flight, detailed in Section \ref{sec:optimisednoreturn}, and is shown in Appendix \ref{sec:appendix_firststage}. 
 After the easterly launch, and first stage pitchover, the SPARTAN is released in an easterly direction, and performs a banking manoeuvre throughout its acceleration so that the heading angle is pointed appropriately close to north-northwest at the release of the third stage rocket. 
\begin{figure}[ht!]
\centering
\includegraphics[width=1\linewidth]{../LODESTAR_FINAL/Results/mode11/SecondStageStandard}
\caption{}
\label{fig:SecondStageStandard}
\end{figure}
\begin{figure}[ht!]
\centering
\includegraphics[width=0.9\linewidth]{../LODESTAR_FINAL/Results/mode11/NetIspStandard}
\caption{}
\label{fig:NetIspStandard}
\end{figure}
During the ascent, the bank angle of the SPARTAN is significantly large and the angle of attack of the SPARTAN is significantly higher than during the maximum payload-to-orbit trajectory with no fly-back, detailed in Section XX.
As a consequence, the net specific impulse of the SPARTAN is decreased during its acceleration, which can be observed in Figure \ref{fig:NetIspStandard}. This higher angle of attack results in no altitude raising manoeuvre in the middle of the acceleration trajectory of the SPARTAN. The SPARTAN no longer flies within the homogeneous region of the specific impulse of the C-REST engines, instead the flight conditions are close to a region where an increase in angle of attack causes a sharp decrease in specific impulse. 
This indicates that at Mach 7 and 8 the angle of attack, and consequently, the allowable bank angle, of the SPARTAN may be being limited by the performance of the C-REST engines. 

The first stage-SPARTAN separation is at a significantly higher altitude and trajectory angle when compared to the optimised trajectory with no SPARTAN fly-back in Section XX. This higher release point allows the first stage to achieve a higher velocity at separation, and causes a large altitude raising manoeuvre at the beginning of the SPARTAN's acceleration, which allows time for the bank angle of the SPARTAN to be increased. 
 The bank angle is initially increased at the maximum change rate to XX$^\circ$ after the first stage-SPARTAN separation, which aids the SPARTAN in decreasing its altitude. As the dynamic pressure of the SPARTAN approaches its maximum, the bank angle stop increasing and the angle of attack is increased, to provide more lift to the SPARTAN to stop its descent. 
As the SPARTAN reaches close to its maximum dynamic pressure at XXs, the angle of attack is reduced, and then the bank angle is increased, up to a maximum of XX. 
 The bank angle stays between XX$^\circ$ - XX$^\circ$ for the duration of the acceleration, until a pull-up is performed at XXs. During the pull-up, the bank angle is steadily reduced at its maximum change rate, until the release of the third stage rocket at 0$^\circ$ bank angle. 

A total fuel mass of XXkg is used during the SPARTAN's acceleration. This reduction in fuel mass used, along with the reduction in net specific impulse due to the higher angle of attack values, reduces the velocity at SPARTAN-third stage separation by XXm/s compared to the maximum payload-to-orbit case with no SPARTAN fly-back. The SPARTAN pulls up to \secondthirdSeparationAltStandard km altitude and \secondthirdSeparationgammaStandard $^\circ$ before SPARTAN-third stage separation, a difference of only XXkm and XX$^\circ$ compared to the maximum payload-to-orbit trajectory without fly-back. 


-doesnt do the hop up to minimise ground distance covered


\begin{figure}[ht!]
\centering
\includegraphics[width=1\linewidth]{../LODESTAR_FINAL/Results/mode11/ThirdStageStandard}
\caption{}
\label{fig:ThirdStageStandard}
\end{figure}


\section{Fly-Back Trajectory}

\begin{figure}[ht!]
	\centering
	\includegraphics[width=1\linewidth]{../LODESTAR_FINAL/Results/mode11/ReturnStandard}
	\caption{}
	\label{fig:ReturnStandard}
\end{figure}

The optimised fly-back trajectory is shown in Figure \ref{fig:ReturnStandard}.
The SPARTAN is shown to be capable of fly-back, using XXkg of fuel, XX\% of the total fuel.
The optimised trajectory has four distinct parts; 1. initial turn, 2. boost phase, 3. hop-glide, and 4. approach. 
 The skips which the SPARTAN exhibits during its return flight are aided by the angle of attack of the SPARTAN, and are consistent with research which has shown that a periodic skipping trajectory increases the downrange distance achievable by hypersonic vehicles both during powered and unpowered flight\cite{Eggers1957,Kanda2007}. 
 
\subsubsection{ Initial Turn}
The SPARTAN separates from the third stage rocket at a bank angle of XX$^\circ$, and then increases the bank angle at its maximum change rate until XXs and XX$^\circ$ bank angle.
The angle of attack is also decreasing during this time, to a minimum of XX, in order to minimise the size of the initial skip. 
 At this point, the altitude of the SPARTAN is decreasing, and the SPARTAN rapidly becomes close to hitting the dynamic pressure of 50kPa. To avoid exceeding this limit, the bank angle is reduced to 67.5$^\circ$ at 164.3s, allowing the vehicle to generate sufficient lift to slow its descent. 

\subsubsection{ Boost Phase}
Soon after the bank angle has reduced to XX$^\circ$, the scramjet engines are ignited. The C-REST engines are powered-on at a point of high potential specific impulse, at a low Mach number, and burn for XXs. The altitude of the SPARTAN is raised during the initial burn, ensuring that the Mach number is kept low for maximum efficiency. The maximum altitude during the burn is limited by the lower dynamic pressure limit of the C-REST engines of 20kPa.  Before the initial burn, both the angle of attack and bank angle of the SPARTAN are decreased. The bank angle of the vehicle is reduced to produce increased lift, so as to increase the altitude of the SPARTAN, while also maximising the specific impulse of the scramjet engines by keeping the angle of attack low. The low angle of attack values decrease the temperature and raise the Mach number at the inlet of the C-REST engines. While these effects partially offset each other\cite{Preller2017}, the temperature increase is more significant, and decreasing the angle of attack has the net effect of increasing the specific impulse of the C-REST engines. This increase in specific impulse is balanced by a decrease in the L/D of the SPARTAN at angle of attack values lower than 4$^\circ$, as illustrated in Figure \ref{fig:LD}. However, the specific impulse has a more significant impact than the L/D during this phase, resulting in the optimised angle of attack being kept low. 
At XXs, the initial burn ends, the angle of attack of the SPARTAN is decreased to XX$^\circ$, and the altitude decreases. As soon as the dynamic pressure is high enough for C-REST engine operation at XXs, the scramjet engines are once again ignited.
This skip raises the altitude of the SPARTAN to XXkm, before it decreases once again. 
The third and last burn is initiated at XXs and lasts until XXs, when the remaining fuel has been depleted. 



\subsubsection{ Approach}

During the unpowered trajectory after the burn phase, the angle of attack is initially controlled so that the skipping trajectory of the SPARTAN is dampened.

Immediately after the third burn phase, the angle of attack is reduced, to XX. This reduction coincides with the ascent portion of the fourth skip, reducing the amount of altitude gained. 
As the zenith of the forth skip is reached, the angle of attack is increased, once again counteracting the skipping manoeuvre. 
This high angle of attack is sustained until XXs at which point, the angle of attack is reduced again significantly, to XX$^\circ$, reducing the size of the fifth skip significantly. At XXs, the angle of attack is again raised to XX$^\circ$, initiating the sixth and last skip. 
It is notable that the angle of attack is raised at this point, as previously in the unpowered portion of the trajectory the angle of attack is being utilised to damped the skipping motion. This indicates that some degree of skipping is desired, and that the angle of attack is being controlled to produce optimally sized skips. 


After the final small skip, the angle of attack is adjusted, so that a gradual, controlled descent is initiated. 
After the skip phase, as the vehicle is approaching Mach 1, the angle of attack is reduced gradually to bring the SPARTAN down to 1km altitude, in a controlled manner. At XXs, the bank angle is increased, in order to perform a final adjustment of the heading angle, to bring the SPARTAN to the desired end location. 
The SPARTAN reaches 1km altitude at XX$^\circ$ trajectory angle and XXm/s velocity, and it is assumed that the SPARTAN is able to perform a landing manoeuvre after this point. 








\section{Design Sensitivity Analysis}

It has been shown that the fly-back of the SPARTAN accelerator has a significant effect on the performance of the rocket-scramjet-rocket launch system, and that the maximum payload-to-orbit optimised trajectory changes significantly to compensate for the additional requirement of successfully returning the SPARTAN stage. The sensitivity of the launch system to changes in the vehicle design may be significantly affected by the fly-back trajectory, and as such, a sensitivity study is conducted with the fly-back of the SPARTAN included. As in Section \ref{sec:sensitivityNoReturn} this sensitivity study varies the following:
\begin{itemize}
	\item Dynamic Pressure
	\item Specific Impulse
	\item SPARTAN Drag
	\item SPARTAN Mass
	\item SPARTAN Fuel Mass
	\item Third Stage Mass
	\item Third Stage Thrust
\end{itemize}
As in Section \ref{sec:sensitivityNoReturn}, the effect of third stage drag is negligible. For this reason, variation in the third stage drag is omitted from this study. 



In addition to investigating the trends in the performance of the launch system, this sensitivity study serves to verify the ability of LODESTAR to generate optimal trajectories with varied vehicle designs, as well as investigating the robustness of the optimised solution.
The launch system is able to successfully place a small satellite in orbit for every varied performance condition which has been tested, while returning the SPARTAN to its initial launch location for landing. 
Every maximum payload-to-orbit optimised trajectory exhibits considerable banking during the SPARTAN's ascent trajectory, as well as a pull-up of the SPARTAN before third stage release. 
In every case the optimised return flight path exhibits an initial turn, boost and approach phase, with multiple skipping manoeuvres. 
However, two aspects of the optimised trajectories vary between cases, exhibiting no clear trend across the sensitivity studies which have been performed; the first stage-SPARTAN release conditions, and the size of the second skip of the return phase. 

\textcolor{red}{I need to make sure this is correct after all solutions are run}

\textcolor{red}{i may be making too big of a deal about this, note similar work done}
The first stage-SPARTAN separation angle and altitude shows no clear trend in any of the sensitivity studies performed, in contrast to the sensitivity study with no fly-back, detailed in Section XX, in which the mass and drag parameters change the first stage release significantly. All of the optimised trajectory solutions show a distinct initial altitude raising manoeuvre, however, their size is inconsistent across optimised trajectory solutions, indicating that this manoeuvre is no longer solely a product of an efficiency trade-off between the first stage pitching and SPARTAN engine efficiency. When fly-back is included and the SPARTAN is banking, this altitude raising manoeuvre allows large bank angles, changing the heading angle of the SPARTAN rapidly, while maintaining low angles of attack for efficiency. 
In the maximum payload-to-orbit optimised trajectories calculated during the sensitivity analysis, it is observed that the trajectory angle at first-second stage release varies significantly between the optimised trajectories, in often unpredictable ways. When the SPARTAN is released at a high trajectory angle, there is a significant amount of time allowed for the bank angle to increase, and the high bank angle is utilised during the descent of the SPARTAN onto the maximum dynamic pressure path, where the trajectory angle is negative and the heading angle changes more rapidly. 
When the trajectory angle at release of the SPARTAN is lower, the first stage is generally XX \textcolor{red}{look at exergy analysis}. A lower release angle results in the SPARTAN banking more, and flying a slightly less efficient trajectory. However, a lower release angle also results in the SPARTAN using its fuel more rapidly, and covering less ground, which results in the fly-back requiring less fuel. 
The trade-off between first stage efficiency and the initial operational efficiency of the SPARTAN appears to be close, and 
for each particular trajectory optimisation one or the other is favoured. 


It is also observed that there are two distinct return trajectory shapes for the return trajectory of the SPARTAN. The more common return trajectory shape has been shown in the preceding section, and consists of three or more large skips to begin the return trajectory. The second trajectory shape exhibits a small second skip, with the first two burns very closely spaced, or combined into one longer burn. During the first two burns, a high bank angle is maintained  when compared to the large skip trajectory shape, however, after the first two burns are completed, the bank angle is reduced more rapidly. 
An example of the second type of return trajectory is shown in Figure XX(appendix). 
During simulations, it was observed that on occasion, the optimal return trajectory type would change as the initial guess or problem setup was altered, with no significant change in the fuel mass used during the return, or the payload-to-orbit capabilities of the launch system. This variability suggests that there is minimal difference between the two shapes of return trajectory. 





\subsection{dynamic pressure}



The maximum dynamic pressure  of the SPARTAN has an increased effect on the maximum payload-to-orbit optimised trajectory which includes the SPARTAN flying-back to the launch site. This is primarily due to the increased maximum dynamic pressure improving the manoeuvring capabilities of the SPARTAN and increasing the acceleration rate during ascent, which leads to a smaller flight time, and less ground coverage, generally reducing the amount of fuel necessary for fly-back. The exception is the fly-back of the 60kPa maximum dynamic pressure trajectory, which uses more fuel than the fly-back of the 55kPa maximum dynamic pressure trajectory. 

Table \ref{tab:qvarreturn} shows a summary of the key parameters of each optimised trajectory, and trajectory comparison plots are shown in Appendix XX. The variation in each trajectory parameter per \% of the dynamic pressure is shown, if there is a clear trend. 



\begin{table}[ht]
\centering
\begin{tabular}{l c c c c c c} 
	\hline \textbf{Trajectory Condition}
	&q40
	&q45
	&q50
	&q55
	&q60
	& /\%
	\\
	\hline \textbf{Payload to Orbit (kg)}
	& \PayloadToOrbitqForty
	& \PayloadToOrbitqFortyFive
	& \PayloadToOrbitqStandard
	& \PayloadToOrbitqFiftyFive
	& \PayloadToOrbitqSixty
	&0.4
	\\
	\textbf{Separation Alt, 1$\rightarrow$2 (km)}
	& \firstsecondSeparationAltqForty
	& \firstsecondSeparationAltqFortyFive
	& \firstsecondSeparationAltqStandard
	& \firstsecondSeparationAltqFiftyFive
	& \firstsecondSeparationAltqSixty
	& -
	\\
	\textbf{Separation v, 1$\rightarrow$2 (m/s)}
	& \firstsecondSeparationvqForty
	& \firstsecondSeparationvqFortyFive
	& \firstsecondSeparationvqStandard
	& \firstsecondSeparationvqFiftyFive
	& \firstsecondSeparationvqSixty
	& -
	\\
	\textbf{Separation $\gamma$, 1$\rightarrow$2 (m/s)}
	& \firstsecondSeparationgammaqForty
	& \firstsecondSeparationgammaqFortyFive
	& \firstsecondSeparationgammaqStandard
	& \firstsecondSeparationgammaqFiftyFive
	& \firstsecondSeparationgammaqSixty
	& -
	\\
	\textbf{1st Stage $\Delta$Exergy (GJ)}
	& \firstdExergyqForty
	& \firstdExergyqFortyFive
	& \firstdExergyqStandard
	& \firstdExergyqFiftyFive
	& \firstdExergyqSixty
	& -
	\\
	\textbf{Separation Alt, 2$\rightarrow$3 (km)}
	& \secondthirdSeparationAltqForty
	& \secondthirdSeparationAltqFortyFive
	& \secondthirdSeparationAltqStandard
	& \secondthirdSeparationAltqFiftyFive
	& \secondthirdSeparationAltqSixty
	& -
	\\
	\textbf{Separation $v$, 2$\rightarrow$3 (m/s)}
	& \secondthirdSeparationvqForty
	& \secondthirdSeparationvqFortyFive
	& \secondthirdSeparationvqStandard
	& \secondthirdSeparationvqFiftyFive
	& \secondthirdSeparationvqSixty
	&2.06
	\\
	\textbf{Separation $\gamma$, 2$\rightarrow$3 (deg)}
	& \secondthirdSeparationgammaqForty
	& \secondthirdSeparationgammaqFortyFive
	& \secondthirdSeparationgammaqStandard
	& \secondthirdSeparationgammaqFiftyFive
	& \secondthirdSeparationgammaqSixty
	& -
	\\
	\textbf{Separation $q$, 2$\rightarrow$3(kPa)}
	& \secondthirdSeparationqqForty
	& \secondthirdSeparationqqFortyFive
	& \secondthirdSeparationqqStandard
	& \secondthirdSeparationqqFiftyFive
	& \secondthirdSeparationqqSixty
	&0.05
	\\
	\textbf{2$^{nd}$ Stage L/D, 2$\rightarrow$3}
	& \secondthirdSeparationLDqForty
	& \secondthirdSeparationLDqFortyFive
	& \secondthirdSeparationLDqStandard
	& \secondthirdSeparationLDqFiftyFive
	& \secondthirdSeparationLDqSixty
	&0.01
	\\
	\textbf{2$^{nd}$ Stage Flight Time (s)}
	& \secondFlightTimeqForty
	& \secondFlightTimeqFortyFive
	& \secondFlightTimeqStandard
	& \secondFlightTimeqFiftyFive
	& \secondFlightTimeqSixty
	&-2.57
	\\
	\textbf{2nd Stage $\Delta$Exergy (GJ)}
	& \seconddExergyqForty
	& \seconddExergyqFortyFive
	& \seconddExergyqStandard
	& \seconddExergyqFiftyFive
	& \seconddExergyqSixty
	&0.04
	\\
	\textbf{2$^{nd}$ Stage Return Fuel (kg)}
	& \returnFuelqForty
	& \returnFuelqFortyFive
	& \returnFuelqStandard
	& \returnFuelqFiftyFive
	& \returnFuelqSixty
	& -
	\\
	\textbf{3$^{rd}$ Stage $t$, $q >$ 5kpa (s)}
	& \thirdqOverFiveqForty
	& \thirdqOverFiveqFortyFive
	& \thirdqOverFiveqStandard
	& \thirdqOverFiveqFiftyFive
	& \thirdqOverFiveqSixty
	& -
	\\
	\textbf{3$^{rd}$ Stage max $\alpha$ (deg)}
	& \thirdmaxAoAqForty
	& \thirdmaxAoAqFortyFive
	& \thirdmaxAoAqStandard
	& \thirdmaxAoAqFiftyFive
	& \thirdmaxAoAqSixty
	&0
	\\
	\textbf{3$^{rd}$ Stage Circularisation v (m/s)}
	& \thirdcircvqForty
	& \thirdcircvqFortyFive
	& \thirdcircvqStandard
	& \thirdcircvqFiftyFive
	& \thirdcircvqSixty
	& -
	\\
	\textbf{3$^{rd}$ Stage Circularisation m (kg)}
	& \thirdcircmqForty
	& \thirdcircmqFortyFive
	& \thirdcircmqStandard
	& \thirdcircmqFiftyFive
	& \thirdcircmqSixty
	&2.85
	\\
	\textbf{3$^{rd}$ Stage Circularisation Energy (GJ)}
	& \thirdcircEnergyqForty
	& \thirdcircEnergyqFortyFive
	& \thirdcircEnergyqStandard
	& \thirdcircEnergyqFiftyFive
	& \thirdcircEnergyqSixty
	&0.19
	\\
	\hline 
\end{tabular} 
\caption{}
\label{tab:qvarreturn}
\end{table}


\subsection{Isp}

The specific impulse of the SPARTAN is varied by $\pm10\%$ in order to assess the sensitivity of the optimised trajectory to the performance of the scramjet engines.  
Increasing the specific impulse by 10\% causes the fuel necessary for fly-back to decrease by -XXkg (-XX\%), while decreasing the specific impulse by 10\% causes the fuel necessary to rise by +XXkg (+XX\%). 
As was observed in Section XX in the maximum payload-to-orbit trajectory with no fly-back, the first stage trajectory is not significantly changed with SPARTAN specific impulse variation, and consequently the first stage-SPARTAN separation conditions are consistent. Following separation, the shape of the SPARTAN's acceleration is very similar with specific impulse variation, including the the pull-up location and acceleration flight times. As with the optimised trajectories with no fly-back, the specific impulse increased increases the velocity at separation and decreases the trajectory angle, which directly increases the velocity of the third stage at circularisation.
Increasing the specific impulse reduces the fuel used for fly-back by only 1.6kg/\%. 
This is due to the additional velocity and fuel usage throughout the trajectory making the SPARTAN change heading angle more slowly, and cover more ground. This greater ground coverage increases the fly-back distance and partially offsets the beneficial effects of the higher acceleration obtained from the increased specific impulse.

While an increase in the specific impulse of the SPARTAN's scramjet engines is significantly beneficial, the sensitivity of the trajectory to specific impulse is decreased by 22.7\% compared to the sensitivity study of the optimised trajectory with no fly-back. 
The specific impulse has a significantly smaller effect on the separation velocity when compared to the sensitivity study with no fly-back. Increasing the specific impulse increases the SPARTAN-third stage separation velocity by only 11.22m/s/\% compared to 13.74m/s/\% without fly-back.  
This smaller effect is due to the lower second stage acceleration time, which reduces the additional velocity which can be imparted to the third stage rocket due to the increased specific impulse. 






\begin{table}[ht]
	\centering
\begin{tabular}{l c c c c c c} 
	\hline \textbf{Trajectory Condition}
	&Isp90
	&Isp95
	&Isp100
	&Isp105
	&Isp110
	& /\%
	\\
	\hline \textbf{Payload to Orbit (kg)}
	& \PayloadToOrbitIspNinety
	& \PayloadToOrbitIspNinetyFive
	& \PayloadToOrbitIspStandard
	& \PayloadToOrbitIspOneHundredFive
	& \PayloadToOrbitIspOneHundredTen
	&1.7
	\\
	\textbf{Separation Alt, 1$\rightarrow$2 (km)}
	& \firstsecondSeparationAltIspNinety
	& \firstsecondSeparationAltIspNinetyFive
	& \firstsecondSeparationAltIspStandard
	& \firstsecondSeparationAltIspOneHundredFive
	& \firstsecondSeparationAltIspOneHundredTen
	& -
	\\
	\textbf{Separation v, 1$\rightarrow$2 (m/s)}
	& \firstsecondSeparationvIspNinety
	& \firstsecondSeparationvIspNinetyFive
	& \firstsecondSeparationvIspStandard
	& \firstsecondSeparationvIspOneHundredFive
	& \firstsecondSeparationvIspOneHundredTen
	& -
	\\
	\textbf{Separation $\gamma$, 1$\rightarrow$2 (m/s)}
	& \firstsecondSeparationgammaIspNinety
	& \firstsecondSeparationgammaIspNinetyFive
	& \firstsecondSeparationgammaIspStandard
	& \firstsecondSeparationgammaIspOneHundredFive
	& \firstsecondSeparationgammaIspOneHundredTen
	& -
	\\
	\textbf{1st Stage $\Delta$Exergy (GJ)}
	& \firstdExergyIspNinety
	& \firstdExergyIspNinetyFive
	& \firstdExergyIspStandard
	& \firstdExergyIspOneHundredFive
	& \firstdExergyIspOneHundredTen
	& -
	\\
	\textbf{Separation Alt, 2$\rightarrow$3 (km)}
	& \secondthirdSeparationAltIspNinety
	& \secondthirdSeparationAltIspNinetyFive
	& \secondthirdSeparationAltIspStandard
	& \secondthirdSeparationAltIspOneHundredFive
	& \secondthirdSeparationAltIspOneHundredTen
	& -
	\\
	\textbf{Separation $v$, 2$\rightarrow$3 (m/s)}
	& \secondthirdSeparationvIspNinety
	& \secondthirdSeparationvIspNinetyFive
	& \secondthirdSeparationvIspStandard
	& \secondthirdSeparationvIspOneHundredFive
	& \secondthirdSeparationvIspOneHundredTen
	&11.13
	\\
	\textbf{Separation $\gamma$, 2$\rightarrow$3 (deg)}
	& \secondthirdSeparationgammaIspNinety
	& \secondthirdSeparationgammaIspNinetyFive
	& \secondthirdSeparationgammaIspStandard
	& \secondthirdSeparationgammaIspOneHundredFive
	& \secondthirdSeparationgammaIspOneHundredTen
	&-0.09
	\\
	\textbf{Separation $q$, 2$\rightarrow$3(kPa)}
	& \secondthirdSeparationqIspNinety
	& \secondthirdSeparationqIspNinetyFive
	& \secondthirdSeparationqIspStandard
	& \secondthirdSeparationqIspOneHundredFive
	& \secondthirdSeparationqIspOneHundredTen
	& -
	\\
	\textbf{2$^{nd}$ Stage L/D, 2$\rightarrow$3}
	& \secondthirdSeparationLDIspNinety
	& \secondthirdSeparationLDIspNinetyFive
	& \secondthirdSeparationLDIspStandard
	& \secondthirdSeparationLDIspOneHundredFive
	& \secondthirdSeparationLDIspOneHundredTen
	& -
	\\
	\textbf{2$^{nd}$ Stage Flight Time (s)}
	& \secondFlightTimeIspNinety
	& \secondFlightTimeIspNinetyFive
	& \secondFlightTimeIspStandard
	& \secondFlightTimeIspOneHundredFive
	& \secondFlightTimeIspOneHundredTen
	& -
	\\
	\textbf{2nd Stage $\Delta$Exergy (GJ)}
	& \seconddExergyIspNinety
	& \seconddExergyIspNinetyFive
	& \seconddExergyIspStandard
	& \seconddExergyIspOneHundredFive
	& \seconddExergyIspOneHundredTen
	&0.24
	\\
	\textbf{2$^{nd}$ Stage Return Fuel (kg)}
	& \returnFuelIspNinety
	& \returnFuelIspNinetyFive
	& \returnFuelIspStandard
	& \returnFuelIspOneHundredFive
	& \returnFuelIspOneHundredTen
	& -
	\\
	\textbf{2nd Stage Return $\Delta$Exergy (GJ)}
	& \returndExergyIspNinety
	& \returndExergyIspNinetyFive
	& \returndExergyIspStandard
	& \returndExergyIspOneHundredFive
	& \returndExergyIspOneHundredTen
	&-0.15
	\\
	\textbf{3$^{rd}$ Stage $t$, $q >$ 5kpa (s)}
	& \thirdqOverFiveIspNinety
	& \thirdqOverFiveIspNinetyFive
	& \thirdqOverFiveIspStandard
	& \thirdqOverFiveIspOneHundredFive
	& \thirdqOverFiveIspOneHundredTen
	& -
	\\
	\textbf{3$^{rd}$ Stage max $\alpha$ (deg)}
	& \thirdmaxAoAIspNinety
	& \thirdmaxAoAIspNinetyFive
	& \thirdmaxAoAIspStandard
	& \thirdmaxAoAIspOneHundredFive
	& \thirdmaxAoAIspOneHundredTen
	& -
	\\
	\textbf{3$^{rd}$ Stage final v (m/s)}
	& \thirdcircvIspNinety
	& \thirdcircvIspNinetyFive
	& \thirdcircvIspStandard
	& \thirdcircvIspOneHundredFive
	& \thirdcircvIspOneHundredTen
	&11.82
	\\
	\textbf{3$^{rd}$ Stage final m (kg)}
	& \thirdcircmIspNinety
	& \thirdcircmIspNinetyFive
	& \thirdcircmIspStandard
	& \thirdcircmIspOneHundredFive
	& \thirdcircmIspOneHundredTen
	& -
	\\
	\textbf{3rd Stage $\Delta$Exergy (GJ)}
	& \thirddExergyIspNinety
	& \thirddExergyIspNinetyFive
	& \thirddExergyIspStandard
	& \thirddExergyIspOneHundredFive
	& \thirddExergyIspOneHundredTen
	& -
	\\
	\hline 
\end{tabular} 

\end{table}

\subsection{Cd}

The coefficient of drag is varied by $\pm$10\% to investigate the effect of variation in SPARTAN design on the performance of the launch system, including the effects on the fly-back of the SPARTAN. 

Increasing the drag coefficient causes the fuel necessary for fly-back to increase by +XXkg (+XX\%). Conversely, decreasing the drag coefficient by 10\% causes the fuel necessary for  fly-back to decrease by -XXkg (-XX\%). 
When the drag is increased (ie. L/D is decreased), the velocity decreases more rapidly during the return flight, and results in the initial burn beginning sooner as Cd is increased. 
As the Cd is increased, the size of the second skip is generally decreased, resulting in a shorter gap between the first and second burns. At 110\% drag, there is only one long initial burn, and two total burns. 



As in the drag sensitivity study with no return, the SPARTAN-third stage separation angle shows a general increase as drag in increased. However, in contrast to the drag variation study with no fly-back, increasing the drag of the SPARTAN shows a clear trend in decreasing the altitude of SPARTAN-third stage separation. 
As well as this, the L/D at SPARTAN-third stage separation shows the opposite trend to the sensitivity study with no return, decreasing rather than increasing as drag is increased. This is due to the angle of attack of the SPARTAN being reduced at separation when return is present, resulting in a more efficient L/D, and being reduced more as the drag is decreased. 
The release altitude and angle of attack serve to initiate the first skip of the return trajectory in a consistent manner, so that the shape of the initial skip is very similar with drag variation. In all cases the angle of attack is reduced to 0$^\circ$ immediately during return to lessen the size of the initial skip, and is then raised to close to the maximum of 10$^\circ$ to prevent the dynamic pressure limit being exceeded. This consistency indicates that it is the control and structural limitations of the SPARTAN which are driving the conditions at SPARTAN-third stage release. 


-the effect of Cd is reduced compared to the no fly-back case

-some of the negative effects of Cd are offset by the tighter trajectory being flown, and the lower ground distance covered


\begin{table}[ht]
	\centering
\begin{tabular}{l c c c c c c} 
	\hline \textbf{Trajectory Condition}
	&Cd90
	&Cd95
	&Cd100
	&Cd105
	&Cd110
	& /\%
	\\
	\hline \textbf{Payload to Orbit (kg)}
	& \PayloadToOrbitCdNinety
	& \PayloadToOrbitCdNinetyFive
	& \PayloadToOrbitCdStandard
	& \PayloadToOrbitCdOneHundredFive
	& \PayloadToOrbitCdOneHundredTen
	&-1.5
	\\
	\textbf{Separation Alt, 1$\rightarrow$2 (km)}
	& \firstsecondSeparationAltCdNinety
	& \firstsecondSeparationAltCdNinetyFive
	& \firstsecondSeparationAltCdStandard
	& \firstsecondSeparationAltCdOneHundredFive
	& \firstsecondSeparationAltCdOneHundredTen
	& -
	\\
	\textbf{Separation v, 1$\rightarrow$2 (m/s)}
	& \firstsecondSeparationvCdNinety
	& \firstsecondSeparationvCdNinetyFive
	& \firstsecondSeparationvCdStandard
	& \firstsecondSeparationvCdOneHundredFive
	& \firstsecondSeparationvCdOneHundredTen
	&-3.66
	\\
	\textbf{Separation $\gamma$, 1$\rightarrow$2 (m/s)}
	& \firstsecondSeparationgammaCdNinety
	& \firstsecondSeparationgammaCdNinetyFive
	& \firstsecondSeparationgammaCdStandard
	& \firstsecondSeparationgammaCdOneHundredFive
	& \firstsecondSeparationgammaCdOneHundredTen
	& -
	\\
	\textbf{1st Stage $\Delta$Exergy (GJ)}
	& \firstdExergyCdNinety
	& \firstdExergyCdNinetyFive
	& \firstdExergyCdStandard
	& \firstdExergyCdOneHundredFive
	& \firstdExergyCdOneHundredTen
	&-0.06
	\\
	\textbf{Separation Alt, 2$\rightarrow$3 (km)}
	& \secondthirdSeparationAltCdNinety
	& \secondthirdSeparationAltCdNinetyFive
	& \secondthirdSeparationAltCdStandard
	& \secondthirdSeparationAltCdOneHundredFive
	& \secondthirdSeparationAltCdOneHundredTen
	&-0.04
	\\
	\textbf{Separation $v$, 2$\rightarrow$3 (m/s)}
	& \secondthirdSeparationvCdNinety
	& \secondthirdSeparationvCdNinetyFive
	& \secondthirdSeparationvCdStandard
	& \secondthirdSeparationvCdOneHundredFive
	& \secondthirdSeparationvCdOneHundredTen
	&-9.5
	\\
	\textbf{Separation $\gamma$, 2$\rightarrow$3 (deg)}
	& \secondthirdSeparationgammaCdNinety
	& \secondthirdSeparationgammaCdNinetyFive
	& \secondthirdSeparationgammaCdStandard
	& \secondthirdSeparationgammaCdOneHundredFive
	& \secondthirdSeparationgammaCdOneHundredTen
	& -
	\\
	\textbf{Separation $q$, 2$\rightarrow$3(kPa)}
	& \secondthirdSeparationqCdNinety
	& \secondthirdSeparationqCdNinetyFive
	& \secondthirdSeparationqCdStandard
	& \secondthirdSeparationqCdOneHundredFive
	& \secondthirdSeparationqCdOneHundredTen
	& -
	\\
	\textbf{2$^{nd}$ Stage L/D, 2$\rightarrow$3}
	& \secondthirdSeparationLDCdNinety
	& \secondthirdSeparationLDCdNinetyFive
	& \secondthirdSeparationLDCdStandard
	& \secondthirdSeparationLDCdOneHundredFive
	& \secondthirdSeparationLDCdOneHundredTen
	&-0.05
	\\
	\textbf{2$^{nd}$ Stage Flight Time (s)}
	& \secondFlightTimeCdNinety
	& \secondFlightTimeCdNinetyFive
	& \secondFlightTimeCdStandard
	& \secondFlightTimeCdOneHundredFive
	& \secondFlightTimeCdOneHundredTen
	& -
	\\
	\textbf{2nd Stage $\Delta$Exergy (GJ)}
	& \seconddExergyCdNinety
	& \seconddExergyCdNinetyFive
	& \seconddExergyCdStandard
	& \seconddExergyCdOneHundredFive
	& \seconddExergyCdOneHundredTen
	&-0.15
	\\
	\textbf{2$^{nd}$ Stage Return Fuel (kg)}
	& \returnFuelCdNinety
	& \returnFuelCdNinetyFive
	& \returnFuelCdStandard
	& \returnFuelCdOneHundredFive
	& \returnFuelCdOneHundredTen
	& -
	\\
	\textbf{2nd Stage Return $\Delta$Exergy (GJ)}
	& \returndExergyCdNinety
	& \returndExergyCdNinetyFive
	& \returndExergyCdStandard
	& \returndExergyCdOneHundredFive
	& \returndExergyCdOneHundredTen
	&0.12
	\\
	\textbf{3$^{rd}$ Stage $t$, $q >$ 5kpa (s)}
	& \thirdqOverFiveCdNinety
	& \thirdqOverFiveCdNinetyFive
	& \thirdqOverFiveCdStandard
	& \thirdqOverFiveCdOneHundredFive
	& \thirdqOverFiveCdOneHundredTen
	& -
	\\
	\textbf{3$^{rd}$ Stage max $\alpha$ (deg)}
	& \thirdmaxAoACdNinety
	& \thirdmaxAoACdNinetyFive
	& \thirdmaxAoACdStandard
	& \thirdmaxAoACdOneHundredFive
	& \thirdmaxAoACdOneHundredTen
	& -
	\\
	\textbf{3$^{rd}$ Stage final v (m/s)}
	& \thirdcircvCdNinety
	& \thirdcircvCdNinetyFive
	& \thirdcircvCdStandard
	& \thirdcircvCdOneHundredFive
	& \thirdcircvCdOneHundredTen
	&-8.51
	\\
	\textbf{3$^{rd}$ Stage final m (kg)}
	& \thirdcircmCdNinety
	& \thirdcircmCdNinetyFive
	& \thirdcircmCdStandard
	& \thirdcircmCdOneHundredFive
	& \thirdcircmCdOneHundredTen
	& -
	\\
	\textbf{3rd Stage $\Delta$Exergy (GJ)}
	& \thirddExergyCdNinety
	& \thirddExergyCdNinetyFive
	& \thirddExergyCdStandard
	& \thirddExergyCdOneHundredFive
	& \thirddExergyCdOneHundredTen
	& -
	\\
	\hline 
\end{tabular} 
\end{table}


\subsection{m SPARTAN}


\begin{table}[ht]
\centering
\begin{tabular}{l c c c c c c} 
	\hline \textbf{Trajectory Condition}
	&m
	&m
	&m
	&m
	&m
	& /\%
	\\
	\hline \textbf{Payload to Orbit (kg)}
	& \PayloadToOrbitmSPARTANNinetyFive
	& \PayloadToOrbitmSPARTANNinetySevenFive
	& \PayloadToOrbitmSPARTANStandard
	& \PayloadToOrbitmSPARTANOneHundredTwoFive
	& \PayloadToOrbitmSPARTANOneHundredFive
	&-1.4
	\\
	\textbf{Separation Alt, 1$\rightarrow$2 (km)}
	& \firstsecondSeparationAltmSPARTANNinetyFive
	& \firstsecondSeparationAltmSPARTANNinetySevenFive
	& \firstsecondSeparationAltmSPARTANStandard
	& \firstsecondSeparationAltmSPARTANOneHundredTwoFive
	& \firstsecondSeparationAltmSPARTANOneHundredFive
	& -
	\\
	\textbf{Separation v, 1$\rightarrow$2 (m/s)}
	& \firstsecondSeparationvmSPARTANNinetyFive
	& \firstsecondSeparationvmSPARTANNinetySevenFive
	& \firstsecondSeparationvmSPARTANStandard
	& \firstsecondSeparationvmSPARTANOneHundredTwoFive
	& \firstsecondSeparationvmSPARTANOneHundredFive
	&-7.84
	\\
	\textbf{Separation $\gamma$, 1$\rightarrow$2 (m/s)}
	& \firstsecondSeparationgammamSPARTANNinetyFive
	& \firstsecondSeparationgammamSPARTANNinetySevenFive
	& \firstsecondSeparationgammamSPARTANStandard
	& \firstsecondSeparationgammamSPARTANOneHundredTwoFive
	& \firstsecondSeparationgammamSPARTANOneHundredFive
	& -
	\\
	\textbf{1st Stage $\Delta$Exergy (GJ)}
	& \firstdExergymSPARTANNinetyFive
	& \firstdExergymSPARTANNinetySevenFive
	& \firstdExergymSPARTANStandard
	& \firstdExergymSPARTANOneHundredTwoFive
	& \firstdExergymSPARTANOneHundredFive
	&-0.09
	\\
	\textbf{Separation Alt, 2$\rightarrow$3 (km)}
	& \secondthirdSeparationAltmSPARTANNinetyFive
	& \secondthirdSeparationAltmSPARTANNinetySevenFive
	& \secondthirdSeparationAltmSPARTANStandard
	& \secondthirdSeparationAltmSPARTANOneHundredTwoFive
	& \secondthirdSeparationAltmSPARTANOneHundredFive
	&-0.06
	\\
	\textbf{Separation $v$, 2$\rightarrow$3 (m/s)}
	& \secondthirdSeparationvmSPARTANNinetyFive
	& \secondthirdSeparationvmSPARTANNinetySevenFive
	& \secondthirdSeparationvmSPARTANStandard
	& \secondthirdSeparationvmSPARTANOneHundredTwoFive
	& \secondthirdSeparationvmSPARTANOneHundredFive
	&-8.52
	\\
	\textbf{Separation $\gamma$, 2$\rightarrow$3 (deg)}
	& \secondthirdSeparationgammamSPARTANNinetyFive
	& \secondthirdSeparationgammamSPARTANNinetySevenFive
	& \secondthirdSeparationgammamSPARTANStandard
	& \secondthirdSeparationgammamSPARTANOneHundredTwoFive
	& \secondthirdSeparationgammamSPARTANOneHundredFive
	& -
	\\
	\textbf{Separation $q$, 2$\rightarrow$3(kPa)}
	& \secondthirdSeparationqmSPARTANNinetyFive
	& \secondthirdSeparationqmSPARTANNinetySevenFive
	& \secondthirdSeparationqmSPARTANStandard
	& \secondthirdSeparationqmSPARTANOneHundredTwoFive
	& \secondthirdSeparationqmSPARTANOneHundredFive
	& -
	\\
	\textbf{2$^{nd}$ Stage L/D, 2$\rightarrow$3}
	& \secondthirdSeparationLDmSPARTANNinetyFive
	& \secondthirdSeparationLDmSPARTANNinetySevenFive
	& \secondthirdSeparationLDmSPARTANStandard
	& \secondthirdSeparationLDmSPARTANOneHundredTwoFive
	& \secondthirdSeparationLDmSPARTANOneHundredFive
	& -
	\\
	\textbf{2$^{nd}$ Stage Flight Time (s)}
	& \secondFlightTimemSPARTANNinetyFive
	& \secondFlightTimemSPARTANNinetySevenFive
	& \secondFlightTimemSPARTANStandard
	& \secondFlightTimemSPARTANOneHundredTwoFive
	& \secondFlightTimemSPARTANOneHundredFive
	& -
	\\
	\textbf{2nd Stage $\Delta$Exergy (GJ)}
	& \seconddExergymSPARTANNinetyFive
	& \seconddExergymSPARTANNinetySevenFive
	& \seconddExergymSPARTANStandard
	& \seconddExergymSPARTANOneHundredTwoFive
	& \seconddExergymSPARTANOneHundredFive
	&0.06
	\\
	\textbf{2$^{nd}$ Stage Return Fuel (kg)}
	& \returnFuelmSPARTANNinetyFive
	& \returnFuelmSPARTANNinetySevenFive
	& \returnFuelmSPARTANStandard
	& \returnFuelmSPARTANOneHundredTwoFive
	& \returnFuelmSPARTANOneHundredFive
	& -
	\\
	\textbf{3$^{rd}$ Stage $t$, $q >$ 5kpa (s)}
	& \thirdqOverFivemSPARTANNinetyFive
	& \thirdqOverFivemSPARTANNinetySevenFive
	& \thirdqOverFivemSPARTANStandard
	& \thirdqOverFivemSPARTANOneHundredTwoFive
	& \thirdqOverFivemSPARTANOneHundredFive
	& -
	\\
	\textbf{3$^{rd}$ Stage max $\alpha$ (deg)}
	& \thirdmaxAoAmSPARTANNinetyFive
	& \thirdmaxAoAmSPARTANNinetySevenFive
	& \thirdmaxAoAmSPARTANStandard
	& \thirdmaxAoAmSPARTANOneHundredTwoFive
	& \thirdmaxAoAmSPARTANOneHundredFive
	& -
	\\
	\textbf{3$^{rd}$ Stage Circularisation v (m/s)}
	& \thirdcircvmSPARTANNinetyFive
	& \thirdcircvmSPARTANNinetySevenFive
	& \thirdcircvmSPARTANStandard
	& \thirdcircvmSPARTANOneHundredTwoFive
	& \thirdcircvmSPARTANOneHundredFive
	& -
	\\
	\textbf{3$^{rd}$ Stage Circularisation m (kg)}
	& \thirdcircmmSPARTANNinetyFive
	& \thirdcircmmSPARTANNinetySevenFive
	& \thirdcircmmSPARTANStandard
	& \thirdcircmmSPARTANOneHundredTwoFive
	& \thirdcircmmSPARTANOneHundredFive
	& -
	\\
	\textbf{3$^{rd}$ Stage $\Delta$Exergy (GJ)}
	& \thirddExergymSPARTANNinetyFive
	& \thirddExergymSPARTANNinetySevenFive
	& \thirddExergymSPARTANStandard
	& \thirddExergymSPARTANOneHundredTwoFive
	& \thirddExergymSPARTANOneHundredFive
	& -
	\\
	\textbf{Total $\Delta$Exergy (GJ)}
	& \totaldExergymSPARTANNinetyFive
	& \totaldExergymSPARTANNinetySevenFive
	& \totaldExergymSPARTANStandard
	& \totaldExergymSPARTANOneHundredTwoFive
	& \totaldExergymSPARTANOneHundredFive
	& -
	\\
	\hline 
\end{tabular} 

\end{table}





The mass of the SPARTAN is varied by $\pm$5\% to investigate the sensitivity of the launch system performance to the structural mass of the second stage with the inclusion of the fly-back of the SPARTAN. 
Varying the structural mass of the SPARTAN yields a change in potential payload-mass to orbit of 1.2kg per \% of structural mass variation. 
The ascent trajectory varies little with variation in the mass of the SPARTAN. The lower velocity of first stage-SPARTAN separation means that when the SPARTAN is heavier, it is flying at lower velocities, which is beneficial for the specific impulse of the C-REST engines. For this reason, the heavier the SPARTAN is, the more acceleration is obtained over the scramjet-powered ascent. 
Increasing the mass of the SPARTAN decreases the velocity at the end of the first stage by XX\%. Increasing the mass also decreases the altitude of first stage-SPARTAN separation by XX\%. 
It was shown previously in Section XX that decreasing the velocity at first stage-SPARTAN separation does not necessarily affect the altitude of separation. The altitude decrease that is observed is likely to be caused by the increased mass of the SPARTAN requiring greater angle of attack to pull out of the initial altitude raising manoeuvre. As the mass of the SPARTAN increases, the angle of attack is raised for XXs longer to pull out of the descent onto a constant dynamic pressure path, and the decrease in altitude partially offsets the need for this manoeuvre. 

The performance of the launch system shows less sensitivity to the mass of the SPARTAN in the presence of fly-back, 1.2kg/\% compared to 1.5kg/\% without SPARTAN fly-back. 
The sensitivity of the SPARTAN-third stage separation velocity is correspondingly lower, -7.66m/s/\% with fly-back compared to -9.54m/s/\% without fly-back. This decreased effect on the separation velocity is due to the magnitude of the separation velocity, which is significantly lower when fly-back is included. This decreased separation velocity means that the SPARTAN is accelerating in a region of greater net specific impulse, so that the mass has less effect on the overall acceleration. 



\subsection{m Fuel}

\begin{table}[ht]
	\centering
\begin{tabular}{l c c c c c c} 
	\hline \textbf{Trajectory Condition}
	&mF90
	&mF95
	&mF100
	&mF105
	&mF110
	& /\%
	\\
	\hline \textbf{Payload to Orbit (kg)}
	& \PayloadToOrbitmFuelNinety
	& \PayloadToOrbitmFuelNinetyFive
	& \PayloadToOrbitmFuelStandard
	& \PayloadToOrbitmFuelOneHundredFive
	& \PayloadToOrbitmFuelOneHundredTen
	&0.7
	\\
	\textbf{Separation Alt, 1$\rightarrow$2 (km)}
	& \firstsecondSeparationAltmFuelNinety
	& \firstsecondSeparationAltmFuelNinetyFive
	& \firstsecondSeparationAltmFuelStandard
	& \firstsecondSeparationAltmFuelOneHundredFive
	& \firstsecondSeparationAltmFuelOneHundredTen
	& -
	\\
	\textbf{Separation v, 1$\rightarrow$2 (m/s)}
	& \firstsecondSeparationvmFuelNinety
	& \firstsecondSeparationvmFuelNinetyFive
	& \firstsecondSeparationvmFuelStandard
	& \firstsecondSeparationvmFuelOneHundredFive
	& \firstsecondSeparationvmFuelOneHundredTen
	&-2.56
	\\
	\textbf{Separation $\gamma$, 1$\rightarrow$2 (m/s)}
	& \firstsecondSeparationgammamFuelNinety
	& \firstsecondSeparationgammamFuelNinetyFive
	& \firstsecondSeparationgammamFuelStandard
	& \firstsecondSeparationgammamFuelOneHundredFive
	& \firstsecondSeparationgammamFuelOneHundredTen
	&0.05
	\\
	\textbf{Separation Alt, 2$\rightarrow$3 (km)}
	& \secondthirdSeparationAltmFuelNinety
	& \secondthirdSeparationAltmFuelNinetyFive
	& \secondthirdSeparationAltmFuelStandard
	& \secondthirdSeparationAltmFuelOneHundredFive
	& \secondthirdSeparationAltmFuelOneHundredTen
	& -
	\\
	\textbf{Separation $v$, 2$\rightarrow$3 (m/s)}
	& \secondthirdSeparationvmFuelNinety
	& \secondthirdSeparationvmFuelNinetyFive
	& \secondthirdSeparationvmFuelStandard
	& \secondthirdSeparationvmFuelOneHundredFive
	& \secondthirdSeparationvmFuelOneHundredTen
	&5.1
	\\
	\textbf{Separation $\gamma$, 2$\rightarrow$3 (deg)}
	& \secondthirdSeparationgammamFuelNinety
	& \secondthirdSeparationgammamFuelNinetyFive
	& \secondthirdSeparationgammamFuelStandard
	& \secondthirdSeparationgammamFuelOneHundredFive
	& \secondthirdSeparationgammamFuelOneHundredTen
	&-0.05
	\\
	\textbf{Separation $q$, 2$\rightarrow$3(kPa)}
	& \secondthirdSeparationqmFuelNinety
	& \secondthirdSeparationqmFuelNinetyFive
	& \secondthirdSeparationqmFuelStandard
	& \secondthirdSeparationqmFuelOneHundredFive
	& \secondthirdSeparationqmFuelOneHundredTen
	& -
	\\
	\textbf{2$^{nd}$ Stage L/D, 2$\rightarrow$3}
	& \secondthirdSeparationLDmFuelNinety
	& \secondthirdSeparationLDmFuelNinetyFive
	& \secondthirdSeparationLDmFuelStandard
	& \secondthirdSeparationLDmFuelOneHundredFive
	& \secondthirdSeparationLDmFuelOneHundredTen
	& -
	\\
	\textbf{2$^{nd}$ Stage Flight Time (s)}
	& \secondFlightTimemFuelNinety
	& \secondFlightTimemFuelNinetyFive
	& \secondFlightTimemFuelStandard
	& \secondFlightTimemFuelOneHundredFive
	& \secondFlightTimemFuelOneHundredTen
	&4.52
	\\
	\textbf{2$^{nd}$ Stage Return Fuel (kg)}
	& \returnFuelmFuelNinety
	& \returnFuelmFuelNinetyFive
	& \returnFuelmFuelStandard
	& \returnFuelmFuelOneHundredFive
	& \returnFuelmFuelOneHundredTen
	& -
	\\
	\textbf{3$^{rd}$ Stage $t$, $q >$ 5kpa (s)}
	& \thirdqOverFivemFuelNinety
	& \thirdqOverFivemFuelNinetyFive
	& \thirdqOverFivemFuelStandard
	& \thirdqOverFivemFuelOneHundredFive
	& \thirdqOverFivemFuelOneHundredTen
	& -
	\\
	\textbf{3$^{rd}$ Stage max $\alpha$ (deg)}
	& \thirdmaxAoAmFuelNinety
	& \thirdmaxAoAmFuelNinetyFive
	& \thirdmaxAoAmFuelStandard
	& \thirdmaxAoAmFuelOneHundredFive
	& \thirdmaxAoAmFuelOneHundredTen
	& -
	\\
	\textbf{3$^{rd}$ Stage final v (m/s)}
	& \thirdcircvmFuelNinety
	& \thirdcircvmFuelNinetyFive
	& \thirdcircvmFuelStandard
	& \thirdcircvmFuelOneHundredFive
	& \thirdcircvmFuelOneHundredTen
	&6.13
	\\
	\textbf{3$^{rd}$ Stage final m (kg)}
	& \thirdcircmmFuelNinety
	& \thirdcircmmFuelNinetyFive
	& \thirdcircmmFuelStandard
	& \thirdcircmmFuelOneHundredFive
	& \thirdcircmmFuelOneHundredTen
	& -
	\\
	\hline 
\end{tabular} 
\end{table}
	
The fuel mass of the SPARTAN is varied by $\pm$10\%, to investigate the sensitivity of the performance of the launch system to variations in the size of the fuel tanks within the SPARTAN. 
The sensitivity of the performance of the launch system to the fuel mass of the SPARTAN is very similar to the sensitivity observed in the study of the launch system with no SPARTAN fly-back. 
The some of the additional fuel must be used for the fly-back of the SPARTAN, decreasing the effectiveness of the fuel added. However, this is counterbalanced by the additional fuel being utilised at lower velocities than in the no-fly back simulations, so that the additional fuel is used in a region where the C-REST engines have a higher specific impulse.

As the fuel mass is increased, the trajectory of the SPARTAN varies similarly to the preceding section, when the mass of the SPARTAN is increased. An increase in the fuel mass of the SPARTAN results in a slightly longer period of high angle of attack required to pull out of its initial descent. The higher fuel mass then causes the SPARTAN to accelerate more slowly. However, the higher fuel mass allows the SPARTAN to accelerate for far longer, resulting in a significantly increased SPARTAN-third stage separation velocity. 

\subsection{m3}

\begin{table}[ht]
	\centering
	\begin{tabular}{l c c c c c c} 
		\hline \textbf{Trajectory Condition}
		&m390
		&m395
		&m3100
		&m3105
		&m3110
		& /\%
		\\
		\hline \textbf{Payload to Orbit (kg)}
		& \PayloadToOrbitmThreeNinety
		& \PayloadToOrbitmThreeNinetyFive
		& \PayloadToOrbitmThreeStandard
		& \PayloadToOrbitmThreeOneHundredFive
		& \PayloadToOrbitmThreeOneHundredTen
		&1.1
		\\
		\textbf{Separation Alt, 1$\rightarrow$2 (km)}
		& \firstsecondSeparationAltmThreeNinety
		& \firstsecondSeparationAltmThreeNinetyFive
		& \firstsecondSeparationAltmThreeStandard
		& \firstsecondSeparationAltmThreeOneHundredFive
		& \firstsecondSeparationAltmThreeOneHundredTen
		& -
		\\
		\textbf{Separation v, 1$\rightarrow$2 (m/s)}
		& \firstsecondSeparationvmThreeNinety
		& \firstsecondSeparationvmThreeNinetyFive
		& \firstsecondSeparationvmThreeStandard
		& \firstsecondSeparationvmThreeOneHundredFive
		& \firstsecondSeparationvmThreeOneHundredTen
		&-5.25
		\\
		\textbf{Separation $\gamma$, 1$\rightarrow$2 (m/s)}
		& \firstsecondSeparationgammamThreeNinety
		& \firstsecondSeparationgammamThreeNinetyFive
		& \firstsecondSeparationgammamThreeStandard
		& \firstsecondSeparationgammamThreeOneHundredFive
		& \firstsecondSeparationgammamThreeOneHundredTen
		& -
		\\
		\textbf{Separation Alt, 2$\rightarrow$3 (km)}
		& \secondthirdSeparationAltmThreeNinety
		& \secondthirdSeparationAltmThreeNinetyFive
		& \secondthirdSeparationAltmThreeStandard
		& \secondthirdSeparationAltmThreeOneHundredFive
		& \secondthirdSeparationAltmThreeOneHundredTen
		& -
		\\
		\textbf{Separation $v$, 2$\rightarrow$3 (m/s)}
		& \secondthirdSeparationvmThreeNinety
		& \secondthirdSeparationvmThreeNinetyFive
		& \secondthirdSeparationvmThreeStandard
		& \secondthirdSeparationvmThreeOneHundredFive
		& \secondthirdSeparationvmThreeOneHundredTen
		&-4.97
		\\
		\textbf{Separation $\gamma$, 2$\rightarrow$3 (deg)}
		& \secondthirdSeparationgammamThreeNinety
		& \secondthirdSeparationgammamThreeNinetyFive
		& \secondthirdSeparationgammamThreeStandard
		& \secondthirdSeparationgammamThreeOneHundredFive
		& \secondthirdSeparationgammamThreeOneHundredTen
		&0.07
		\\
		\textbf{Separation $q$, 2$\rightarrow$3(kPa)}
		& \secondthirdSeparationqmThreeNinety
		& \secondthirdSeparationqmThreeNinetyFive
		& \secondthirdSeparationqmThreeStandard
		& \secondthirdSeparationqmThreeOneHundredFive
		& \secondthirdSeparationqmThreeOneHundredTen
		&-0.09
		\\
		\textbf{2$^{nd}$ Stage L/D, 2$\rightarrow$3}
		& \secondthirdSeparationLDmThreeNinety
		& \secondthirdSeparationLDmThreeNinetyFive
		& \secondthirdSeparationLDmThreeStandard
		& \secondthirdSeparationLDmThreeOneHundredFive
		& \secondthirdSeparationLDmThreeOneHundredTen
		& -
		\\
		\textbf{2$^{nd}$ Stage Flight Time (s)}
		& \secondFlightTimemThreeNinety
		& \secondFlightTimemThreeNinetyFive
		& \secondFlightTimemThreeStandard
		& \secondFlightTimemThreeOneHundredFive
		& \secondFlightTimemThreeOneHundredTen
		& -
		\\
		\textbf{2$^{nd}$ Stage Return Fuel (kg)}
		& \returnFuelmThreeNinety
		& \returnFuelmThreeNinetyFive
		& \returnFuelmThreeStandard
		& \returnFuelmThreeOneHundredFive
		& \returnFuelmThreeOneHundredTen
		& -
		\\
		\textbf{3$^{rd}$ Stage $t$, $q >$ 5kpa (s)}
		& \thirdqOverFivemThreeNinety
		& \thirdqOverFivemThreeNinetyFive
		& \thirdqOverFivemThreeStandard
		& \thirdqOverFivemThreeOneHundredFive
		& \thirdqOverFivemThreeOneHundredTen
		& -
		\\
		\textbf{3$^{rd}$ Stage max $\alpha$ (deg)}
		& \thirdmaxAoAmThreeNinety
		& \thirdmaxAoAmThreeNinetyFive
		& \thirdmaxAoAmThreeStandard
		& \thirdmaxAoAmThreeOneHundredFive
		& \thirdmaxAoAmThreeOneHundredTen
		& -
		\\
		\textbf{3$^{rd}$ Stage final v (m/s)}
		& \thirdcircvmThreeNinety
		& \thirdcircvmThreeNinetyFive
		& \thirdcircvmThreeStandard
		& \thirdcircvmThreeOneHundredFive
		& \thirdcircvmThreeOneHundredTen
		&-7.82
		\\
		\textbf{3$^{rd}$ Stage final m (kg)}
		& \thirdcircmmThreeNinety
		& \thirdcircmmThreeNinetyFive
		& \thirdcircmmThreeStandard
		& \thirdcircmmThreeOneHundredFive
		& \thirdcircmmThreeOneHundredTen
		&23.37
		\\
		\hline 
	\end{tabular} 
\end{table}

The mass of the third stage rocket is varied by $\pm$10\% to investigate the sensitivity of the launch system to variations in the internal mass of the third stage. The internal mass which is varied is a combination of the fuel, structural and payload mass of the third stage. The structural mass held at 9\% of the total, non-TPS mass, and the remaining mass varied is the fuel and payload.

The sensitivity of the performance of the launch system to the mass of the third stage rocket is much lower when the fly-back of the SPARTAN is included in the trajectory optimisation. 
While the sensitivity of the third stage velocity and mass at circularisation to variations in the third stage mass is similar for the studies with and without fly-back, the release of the third stage from the SPARTAN is at considerably lower velocity when the fly-back of the SPARTAN is included, XX compared to XX for the baseline cases. 
The exponential nature of the circularisation mass change, (exp(delta v)) in Equation XX, means that the velocity decrease at circularisation caused by increasing the third stage mass requires more fuel mass to overcome at lower total velocity. This causes the increased mass of the third stage to contribute less to payload as the total velocity decreases. 


-\textcolor{red}{note exponential velocity dependence vs linear mass dependence}

\subsection{T3}

\textcolor{red}{re-do simulationto change mass flow as well...}

The thrust of the third stage rocket is varied by $\pm$10\% to investigate the sensitivity of the optimal maximum payload-to-orbit trajectory including the fly-back of the SPARTAN. 
The thrust of the third stage has the most significant effect on the payload-to-orbit capability of the launch system, with a sensitivity of 2.6kg per \% thrust variation. 
The sensitivity of the optimal trajectory with SPARTAN fly-back to the third stage thrust is very similar to that observed in Section XX with no SPARTAN fly-back. This indicates that the fly-back does not considerably effect the sensitivity to third stage thrust, and that the third stage thrust has a consistent magnitude of effect at lower separation velocities. 


 As the thrust of the third stage is increased, the third stage rocket is released at lower altitudes, affecting the return trajectory significantly, shown in Figure XX. A lower altitude of release results in a significantly smaller initial skip, which in turn results in the scramjet engines being ignited earlier to avoid exceeding the maximum dynamic pressure limit. The shape of the return trajectory of the 105\% and 110\% thrust simulations are particularly affected, with the 105\% case igniting only twice, with a long initial ignition, and the 110\% case exhibiting  small, rapid skips with far earlier ignitions than the lower third stage thrust simulations. 
This variation in the return trajectory does not have any detrimental effect on the fuel used during the return flight, but indicates that the shape of the optimal trajectory changes significantly with the SPARTAN-third stage release conditions.   




\begin{table}[ht]
	\centering
	\begin{tabular}{l c c c c c c} 
		\hline \textbf{Trajectory Condition}
		&T390
		&T395
		&T3100
		&T3105
		&T3110
		& /\%
		\\
		\hline \textbf{Payload to Orbit (kg)}
		& \PayloadToOrbitTThreeNinety
		& \PayloadToOrbitTThreeNinetyFive
		& \PayloadToOrbitTThreeStandard
		& \PayloadToOrbitTThreeOneHundredFive
		& \PayloadToOrbitTThreeOneHundredTen
		&2.6
		\\
		\textbf{Separation Alt, 1$\rightarrow$2 (km)}
		& \firstsecondSeparationAltTThreeNinety
		& \firstsecondSeparationAltTThreeNinetyFive
		& \firstsecondSeparationAltTThreeStandard
		& \firstsecondSeparationAltTThreeOneHundredFive
		& \firstsecondSeparationAltTThreeOneHundredTen
		& -
		\\
		\textbf{Separation v, 1$\rightarrow$2 (m/s)}
		& \firstsecondSeparationvTThreeNinety
		& \firstsecondSeparationvTThreeNinetyFive
		& \firstsecondSeparationvTThreeStandard
		& \firstsecondSeparationvTThreeOneHundredFive
		& \firstsecondSeparationvTThreeOneHundredTen
		& -
		\\
		\textbf{Separation $\gamma$, 1$\rightarrow$2 (m/s)}
		& \firstsecondSeparationgammaTThreeNinety
		& \firstsecondSeparationgammaTThreeNinetyFive
		& \firstsecondSeparationgammaTThreeStandard
		& \firstsecondSeparationgammaTThreeOneHundredFive
		& \firstsecondSeparationgammaTThreeOneHundredTen
		& -
		\\
		\textbf{Separation Alt, 2$\rightarrow$3 (km)}
		& \secondthirdSeparationAltTThreeNinety
		& \secondthirdSeparationAltTThreeNinetyFive
		& \secondthirdSeparationAltTThreeStandard
		& \secondthirdSeparationAltTThreeOneHundredFive
		& \secondthirdSeparationAltTThreeOneHundredTen
		&-0.27
		\\
		\textbf{Separation $v$, 2$\rightarrow$3 (m/s)}
		& \secondthirdSeparationvTThreeNinety
		& \secondthirdSeparationvTThreeNinetyFive
		& \secondthirdSeparationvTThreeStandard
		& \secondthirdSeparationvTThreeOneHundredFive
		& \secondthirdSeparationvTThreeOneHundredTen
		&5.88
		\\
		\textbf{Separation $\gamma$, 2$\rightarrow$3 (deg)}
		& \secondthirdSeparationgammaTThreeNinety
		& \secondthirdSeparationgammaTThreeNinetyFive
		& \secondthirdSeparationgammaTThreeStandard
		& \secondthirdSeparationgammaTThreeOneHundredFive
		& \secondthirdSeparationgammaTThreeOneHundredTen
		&-0.24
		\\
		\textbf{Separation $q$, 2$\rightarrow$3(kPa)}
		& \secondthirdSeparationqTThreeNinety
		& \secondthirdSeparationqTThreeNinetyFive
		& \secondthirdSeparationqTThreeStandard
		& \secondthirdSeparationqTThreeOneHundredFive
		& \secondthirdSeparationqTThreeOneHundredTen
		&0.54
		\\
		\textbf{2$^{nd}$ Stage L/D, 2$\rightarrow$3}
		& \secondthirdSeparationLDTThreeNinety
		& \secondthirdSeparationLDTThreeNinetyFive
		& \secondthirdSeparationLDTThreeStandard
		& \secondthirdSeparationLDTThreeOneHundredFive
		& \secondthirdSeparationLDTThreeOneHundredTen
		&0.06
		\\
		\textbf{2$^{nd}$ Stage Flight Time (s)}
		& \secondFlightTimeTThreeNinety
		& \secondFlightTimeTThreeNinetyFive
		& \secondFlightTimeTThreeStandard
		& \secondFlightTimeTThreeOneHundredFive
		& \secondFlightTimeTThreeOneHundredTen
		& -
		\\
		\textbf{2$^{nd}$ Stage Return Fuel (kg)}
		& \returnFuelTThreeNinety
		& \returnFuelTThreeNinetyFive
		& \returnFuelTThreeStandard
		& \returnFuelTThreeOneHundredFive
		& \returnFuelTThreeOneHundredTen
		& -
		\\
		\textbf{3$^{rd}$ Stage $t$, $q >$ 5kpa (s)}
		& \thirdqOverFiveTThreeNinety
		& \thirdqOverFiveTThreeNinetyFive
		& \thirdqOverFiveTThreeStandard
		& \thirdqOverFiveTThreeOneHundredFive
		& \thirdqOverFiveTThreeOneHundredTen
		&1.23
		\\
		\textbf{3$^{rd}$ Stage max $\alpha$ (deg)}
		& \thirdmaxAoATThreeNinety
		& \thirdmaxAoATThreeNinetyFive
		& \thirdmaxAoATThreeStandard
		& \thirdmaxAoATThreeOneHundredFive
		& \thirdmaxAoATThreeOneHundredTen
		&-0.01
		\\
		\textbf{3$^{rd}$ Stage final v (m/s)}
		& \thirdcircvTThreeNinety
		& \thirdcircvTThreeNinetyFive
		& \thirdcircvTThreeStandard
		& \thirdcircvTThreeOneHundredFive
		& \thirdcircvTThreeOneHundredTen
		&128.98
		\\
		\textbf{3$^{rd}$ Stage final m (kg)}
		& \thirdcircmTThreeNinety
		& \thirdcircmTThreeNinetyFive
		& \thirdcircmTThreeStandard
		& \thirdcircmTThreeOneHundredFive
		& \thirdcircmTThreeOneHundredTen
		&-62.18
		\\
		\hline 
	\end{tabular} 
\end{table}



\subsection{Comparison of Sensitivities}

-check this (from paper)
These results indicate that the aerodynamic performance of the SPARTAN has significantly more impact than the efficiency of the scramjet engines on the optimised fly-back trajectory. During the fly-back trajectory, the specific impulse is effecting performance only whilst the scramjet engines are operating, compared with the aerodynamics of the vehicle, which effect performance throughout the trajectory. This suggests that, for maximum fly-back performance, the aerodynamic performance should be given preference over engine efficiency in the design of fly-back hypersonic accelerators. 

\begin{figure}[th]
\centering
\includegraphics[width=0.8\linewidth]{figures/6_FlyBack/BarChart}
\caption{Sensitivity of design parameters. \textcolor{red}{include lines showing sensitivities without flyback}}
\label{fig:BarChart}
\end{figure}




\section{Sonic Boom Ground Effects}

The flight of a hypersonic vehicle has the potential to create significant overpressures on the ground due to sonic booms[CITEXX]. Even when the vehicle is flying at high altitudes, the overpressures on the ground may still be large enough to have detrimental effects on any populated areas being overflown. The overpressure from sonic booms can cause significant annoyance to the populace, or in more extreme cases, long term damage to building structures or peoples health. 
When the SPARTAN is launched to a sun synchronous orbit from the Equatorial Launch Australia launch site, it flies over a significant portion of Papua. Fortunately, Papua is sparsely populated, and the number of towns flown over by the SPARTAN will be low, however the effects on these population centres may still be significant. Although it is flying at high altitude, the SPARTAN is flying at hypersonic speeds, and creates significant sonic boom effects. In order to assess the impact of the SPARTAN's flight, the magnitude of the overpressure from its sonic booms must be calculated. 
\begin{figure}[ht]
	\centering
	\includegraphics[width=0.9\linewidth]{../LODESTAR_FINAL/Results/mode11/OverPressureStandard}
	\caption{}
	\label{fig:OverPressureStandard}
\end{figure}

The sonic boom overpressures are estimated using the 'first cut' estimation technique [CITEXX]. This estimation technique can approximate sonic boom overpressures moderately well, and is useful as a first approximation to the sonic boom overpressures generated by an aerospace vehicle. The overpressures generated by the SPARTAN are calculated over its trajectory, shown in Figure \ref{fig:OverPressureStandard}. It is found that overpressures of up to 375.3Pa occur during flight over land during the maximum payload-to-orbit trajectory of the SPARTAN. These overpressures have a low but significant probability of causing cosmetic damage to structures (~1.5\% for plaster and ~0.4\% for glass)[CITEXX-below]. In addition, overpressures of these magnitudes have been rated as unacceptably annoying to the majority populace being overflown, as shown in Figure \ref{fig:OverPressureResponse}. 
These overpressures indicate that overflight of populated areas may not be reasonable for the SPARTAN flying its maximum payload-to-orbit trajectory. 


%http://www.dtic.mil/dtic/tr/fulltext/u2/a028512.pdf





\section{Alternate Launch Locations}

A sun synchronous orbit may be attained in both northerly, and southerly directions. An alternate southerly launch is investigated for the rocket-scramjet-rocket launch system, in the case that flight over Papua is not possible. 


\begin{figure}[th]
\centering
\includegraphics[width=1\linewidth]{../LODESTAR_FINAL/Results/mode01/GroundTrackAlternate}
\caption{}
\label{fig:GroundTrackAlternate}
\end{figure}

\begin{table}
	\centering
\begin{tabular}{l c} 
	\hline \textbf{Trajectory Condition}
	& Alternate
	\\
	\hline \textbf{Payload to Orbit (kg)}
	& \PayloadToOrbitAlternate
	\\
	\textbf{Separation Alt, 1$\rightarrow$2 (km)}
	& \firstsecondSeparationAltAlternate
	\\
	\textbf{Separation v, 1$\rightarrow$2 (m/s)}
	& \firstsecondSeparationvAlternate
	\\
	\textbf{Separation $\gamma$, 1$\rightarrow$2 (m/s)}
	& \firstsecondSeparationgammaAlternate
	\\
	\textbf{Separation Alt, 2$\rightarrow$3 (km)}
	& \secondthirdSeparationAltAlternate
	\\
	\textbf{Separation $v$, 2$\rightarrow$3 (m/s)}
	& \secondthirdSeparationvAlternate
	\\
	\textbf{Separation $\gamma$, 2$\rightarrow$3 (deg)}
	& \secondthirdSeparationgammaAlternate
	\\
	\textbf{Separation $q$, 2$\rightarrow$3(kPa)}
	& \secondthirdSeparationqAlternate
	\\
	\textbf{2$^{nd}$ Stage L/D, 2$\rightarrow$3}
	& \secondthirdSeparationLDAlternate
	\\
	\textbf{2$^{nd}$ Stage Flight Time (s)}
	& \secondFlightTimeAlternate
	\\
	\textbf{3$^{rd}$ Stage $t$, $q >$ 5kpa (s)}
	& \thirdqOverFiveAlternate
	\\
	\textbf{3$^{rd}$ Stage max $\alpha$ (deg)}
	& \thirdmaxAoAAlternate
	\\
	\textbf{3$^{rd}$ Stage final v (m/s)}
	& \thirdcircvAlternate
	\\
	\textbf{3$^{rd}$ Stage final m (kg)}
	& \thirdcircmAlternate
	\\
	\hline 
\end{tabular} 
\end{table}

\section{Summary}
The fly-back trajectory of the SPARTAN hypersonic vehicle is investigated, from separation at 7.7$^\circ$S,145.0$^\circ$E to landing at 15.3$^\circ$S,144.9$^\circ$E, corresponding to a near 180$^\circ$ turn and a fly-back of 878km. The aerodynamics of the SPARTAN are calculated using CART3D, an inviscid CFD package, over the range of Mach numbers and angle of attack values of flight. The optimal trajectory of the SPARTAN is calculated, to fly-back to the initial launch position with minimum fuel. The optimal trajectory is calculated using the launch vehicle optimal control program LODESTAR. It is found that the SPARTAN is capable of returning to its initial launch position, using 166.0kg of fuel. The optimal trajectory terminates when SPARTAN reaches 200m altitude at a velocity of 119.8m/s. After this point, it is assumed that the SPARTAN lands on a traditional runway, at similar conditions to the space shuttle.  
This result indicates that the fly-back of a hypersonic launch vehicle from high velocity separation at a Mach number greater than nine, returning to its initial launch site using scramjet hypersonic airbreathing engines, is feasible. This fly-back to the the original launch site is a crucial component for low cost access-to-space using scramjets. 

The coefficient of drag of the SPARTAN and specific impulse of the scramjet engines were independently varied by $\pm10\%$ and the new optimal trajectories calculated to assess the robustness of the fly-back trajectory to uncertainties in vehicle aerodynamics and scramjet performance. It was found that a $\pm10\%$ variation in $C_D$ results in a +31.0\% or -34.9\% variation in fuel mass burned during fly-back, while a $\pm10\%$ variation in $I_{SP}$ results in a much smaller variation of -6.9\% or +13.8\%. These results indicate that the aerodynamics of a fly-back hypersonic accelerator are much more significant to the fly-back fuel usage than the performance of the scramjet engine. 

