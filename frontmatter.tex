% frontmatter.tex





\thispagestyle{empty}
\begin{center}
	\centering
  \includegraphics{figures/uq_logo}

  \vspace{50pt}

  \textbf{\Large Trajectory Optimisation of a Partially-Reusable Rocket-Scramjet-Rocket Small Satellite Launch System}

  \vspace{25pt}

  {\large Sholto O. Forbes-Spyratos }
    
    \vspace{5pt}

  {\large B.Eng. (Mechanical and Aerospace) (Hons. I) \& B.Sc. (Physics)}


  \vspace{60pt}
                                                                                                                                                                                                                                                                                                                                                                                                                                                                                                                                                                                                                                                                                                                                                                                                                                                                                                                                                                                                                                                 
  \vfill

  {\large A thesis submitted for the degree of Doctor of Philosophy at\\ The University of Queensland in 2018}

  \vspace{20pt}

  School of Mechanical Engineering
  
  \textit{Centre for Hypersonics}

  \vfill
\end{center}

\cleardoublepage

\section*{Abstract}

 The small satellite industry is expanding rapidly, driving a need for dedicated, cost effective, small satellite launchers. The small satellite launchers currently under development are mostly designed to be disposable,  contrary to the design trends of larger launch systems, which are moving towards reusability. These contradictory design philosophies are due to the unfavourable scaling of the systems necessary for reusability, which are not cost effective for small sized launch systems. 
 A possible way to introduce effective reusability to small scale launch systems is to incorporate airbreathing engines within the launch system design. Scramjets are particularly suitable for launch system integration, due to their high Mach number operation and wide operational regime. 
 Because of these advantages, work is ongoing at The University of Queensland to develop a small satellite launch system incorporating a scramjet accelerator. This launcher is a rocket-scramjet-rocket, three stage system, designed to be partially reusable. 
 The trajectory simulations of this launch system have previously been designed assuming that flying the scramjet-powered stage at its maximum dynamic pressure will maximise the efficiency of the scramjet engines, and that this will benefit the performance of the launch system. 
 However, there are complex trade-offs between the rocket and airbreathing stages, which must be accounted for within a partially-airbreathing launch system. The rocket stages perform significantly better at high altitudes, where the efficiency loss due to the drag and atmospheric pressure are diminished. Conversely, the airbreathing stage requires high density air to operate effectively, and will generally perform better at low altitudes. 
 This work develops an optimal trajectory profile for a rocket-scramjet-rocket, three stage launch system, determining the flight path which maximises the payload-to-orbit capabilities of the launch system. 
 
 Significant work has previously been carried out on the design and shape optimisation of the scramjet-powered stage of the launch system, designated the SPARTAN. 
 However, the first stage has not been designed, and the third stage previously used an Aerojet Rocketdyne RL-10-3A motor, which is a costly, pump-fed motor. 
 A first stage rocket is developed, based on a Falcon-1e scaled down lengthwise to 8.5m, an appropriate size to accelerate the SPARTAN to the minimum operable conditions of the scramjet engines. The third stage rocket is redesigned, to be powered by a cost effective SpaceX Kestrel upper stage motor. 
 The aerodynamics of the first stage and the SPARTAN are simulated using computational fluid dynamics, to produce accurate aerodynamic databases. The aerodynamics of the control surfaces of the SPARTAN are simulated, and trimmed aerodynamic databases are created.   
 The aerodynamics of the third stage are modelled using Missile Datcom, and propulsion models are developed for all three stages. The aerodynamic and performance models are used to create an accurate six degree of freedom simulation of the launch system. 
 
 A package is developed to calculate the maximum payload-to-orbit trajectory of the rocket-scramjet-rocket launch system, designated LODESTAR, which uses optimal control theory to design flight paths. For this, LODESTAR utilises GPOPS-2, a pseudospectral method optimal control software. LODESTAR configures GPOPS-2 to calculate optimised trajectory profiles, and provides the dynamic simulation of each vehicle, along with tools to verify the optimised solutions produced by GPOPS-2. 
 
 The launch trajectory is initially simulated assuming that the SPARTAN lands at some position downrange. A launch trajectory is simulated in which the SPARTAN flies at maximum dynamic pressure, for use as a reference and verification case. In order to reach separation conditions which allow the SPARTAN to fly at its maximum dynamic pressure for the duration of its trajectory, the first stage must pitch rapidly, flying at negative angle of attack. The SPARTAN then flies at close to horizontal flight, releasing the third stage at a low trajectory angle, after which the third stage must spend a considerable amount of time in-atmosphere. This trajectory achieves a payload-to-orbit of \PayloadToOrbitConstqNoReturn kg.  
 
The maximum payload-to-orbit trajectory of the launch system is calculated, and is found to differ significantly from the trajectory in which the SPARTAN flies at its maximum dynamic pressure for the duration of its acceleration. 
 The SPARTAN is found to deviate from its maximum dynamic pressure at both stage separation points, and for a segment in the middle of its trajectory.
 Higher first stage-SPARTAN and SPARTAN-third stage separation points result in the efficiency of the SPARTAN reducing, but increase the efficiency of the rocket stages, improving the overall efficiency of the system. 
  Additionally, an altitude raising manoeuvre is performed in a region where the specific impulse of the scramjet engines is relatively homogeneous with varied flight conditions, resulting in a very small performance increase. Overall, flying an optimal trajectory increases the payload-to-orbit of the system launching to sun synchronous orbit to \PayloadToOrbitStandardNoReturn kg, an increase of 19.5\% compared to a trajectory in which the SPARTAN flies at maximum dynamic pressure.  
 
 
 The fly-back of the SPARTAN is included within the trajectory optimisation, and a maximum payload-to-orbit flight path is simulated. 
 It is found that the SPARTAN must ignite its scramjet engines during its return flight, using fuel and causing the fly-back to become an important consideration in the optimal trajectory shape. When the fly-back is included, the first stage pitches towards the east, although the final orbital inclination is a polar sun synchronous orbit. The SPARTAN banks heavily throughout its acceleration to manoeuvre to polar inclination, a manoeuvre which decreases the performance of the SPARTAN, but also reduces the amount of fuel used during fly-back, for a net performance gain. 
The fly-back is found to exhibit multiple `skipping' manoeuvres. These skipping manoeuvres serve to increase the glide range of the SPARTAN, minimising the fuel necessary during the return flight. Additionally, the scramjet engines are powered on at the trough of the first three skips, which allows the scramjet engines to ignite at the points of highest possible specific impulse. In total, 17.2\% of the SPARTAN's fuel mass is used during the return flight, and the launch system is able to deliver \PayloadToOrbitStandard kg of payload to sun synchronous orbit while successfully returning the SPARTAN to its initial launch site.  

A study is conducted to quantify the sensitivity of the launch system to variations in key design parameters. The behaviour of the maximum payload-to-orbit trajectory, both with and without SPARTAN fly-back, is investigated as the physical characteristics of the launch system are modified. 
The sensitivities of coupled design parameters are compared, to quantify their relative impact on the performance of the launch system. The magnitudes of these relative impacts are assessed, to indicate the design trade-offs which will produce an increase in the launch system performance. Importantly, it is found that if a reduction in the maximum dynamic pressure of the SPARTAN by -1kPa reduces the structural and thermal protection mass of the SPARTAN by greater than -26.5kg (or -28.4kg with SPARTAN fly-back), then the performance of the launch system will improve. 

 

\clearpage
\section*{Declaration by author}

  This thesis is composed of my original work, and contains no material previously published or written by another person except where due reference has been made in the text. I have clearly stated the contribution by others to jointly-authored works that I have included in my thesis.

  I have clearly stated the contribution of others to my thesis as a whole, including statistical assistance, survey design, data analysis, significant technical procedures, professional editorial advice, and any other original research work used or reported in my thesis. The content of my thesis is the result of work I have carried out since the commencement of my research higher degree candidature and does not include a substantial part of work that has been submitted to qualify for the award of any other degree or diploma in any university or other tertiary institution. I have clearly stated which parts of my thesis, if any, have been submitted to qualify for another award.

  I acknowledge that an electronic copy of my thesis must be lodged with the University Library and, subject to the policy and procedures of The University of Queensland, the thesis be made available for research and study in accordance with the Copyright Act 1968 unless a period of embargo has been approved by the Dean of the Graduate School.

  I acknowledge that copyright of all material contained in my thesis resides with the copyright holder(s) of that material. Where appropriate I have obtained copyright permission from the copyright holder to reproduce material in this thesis.

\clearpage
\section*{Publications During Candidature}

\subsection*{Journal papers}

\noindent\AtNextCite{\defcounter{maxnames}{99}}\fullcite{ForbesSpyratos2018}\\


\subsection*{Conference papers}

\noindent\AtNextCite{\defcounter{maxnames}{99}}\fullcite{ForbesSpyratos2017}\\

\noindent\AtNextCite{\defcounter{maxnames}{99}}\fullcite{Forbes2018a}\\

\noindent\AtNextCite{\defcounter{maxnames}{99}}\fullcite{chai2017}\\

\section*{Publications Included in This Thesis}

This thesis comprises partly of publications, as allowed by University of Queensland Policy PPL 4.60.07. The papers that have been included have all been published in peer reviewed journals at the time of submission. 

\vspace{\baselineskip}
\AtNextCite{\defcounter{maxnames}{99}}\fullcite{ForbesSpyratos2018}

\begin{center}
  \begin{tabular}{ll}
    \toprule
    Contributor   & Contribution \\
    \midrule
    Sholto O. Forbes-Spyratos             
                                  & Conception and Design (85\%)\\
                                  & Performed simulations (100\%)\\
                                  & Analysis of results (90\%)\\
                                  & Wrote and edited paper (85\%)\\
    \midrule
    Ingo H. Jahn            
                                  & Conception and Design (5\%)\\
                                  & Analysis of results (5\%)\\
                                  & Wrote and edited paper (7.5\%)\\
                                  
        \midrule
        Michael P. Kearney            
								      & Conception and Design (5\%)\\
								      & Wrote and edited paper (7.5\%)\\                       

    \midrule
    Michael K. Smart              
                                  & Conception and Design (5\%)\\
                                  & Analysis of results (5\%)\\
                                  & Wrote and edited paper (5\%)\\
    \bottomrule
  \end{tabular}
\end{center}



\section*{Contributions by Others to the Thesis}

The model of the Baseline SPARTAN was provided for this work by Dr. Dawid Preller and Mr. Joseph Chai, including mass properties, dimensions, and CAD models. The CRESTM10 scramjet engine database was provided for this study by Prof. Michael Smart, consisting of tabulated performance data over a range of inlet conditions. The viscous correction incorporated into the SPARTAN's aerodynamic calculations was performed by Mr. Alexander Ward, and provided for this study in the form of a tabulated aerodynamic database. 

\section*{Statement of Parts of the Thesis Submitted to Qualify for the Award of Another Degree}

None.

\clearpage

\section*{} 
  For Kaitlin, with love to my family and friends and utmost gratitude to my advisors: Ingo Jahn, Michael Kearney, and Michael Smart. 

      \vspace*{\fill}
\begin{flushright}
	\textit{I'm trying to find a way off this planet.}
	
Rocket Raccoon
\end{flushright}
\clearpage
\subsection*{Keywords}
  airbreathing propulsion, scramjets, hypersonics, access-to-space, small satellite launchers, airbreathing launch systems

\subsection*{Australian and New Zealand Standard Research Classification (ANZSRC)}

  ANZSRC code: 090107 Hypersonic Propulsion and Hypersonic Aerodynamics, 20\% \newline
 ANZSRC code: 090106 Flight Dynamics, 10\% \newline
 ANZSRC code: 090108, Satellite, Space Vehicle and Missile Design and Testing, 25\% \newline
ANZSRC code: 090104 Aircraft Performance and Flight Control Systems, 15\% \newline
ANZSRC code: 010303 Optimisation, 30\% \newline

\subsection*{Fields of Research (FoR) Classification}

  FoR code: 0901, Aerospace Engineering, 100\%

\tableofcontents

\listoffigures
\addcontentsline{toc}{chapter}{List of figures}
\listoftables
\addcontentsline{toc}{chapter}{List of tables}
\printnomenclature