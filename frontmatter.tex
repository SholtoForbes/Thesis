% frontmatter.tex





\thispagestyle{empty}
\begin{center}
	\centering
  \includegraphics{figures/uq_logo}

  \vspace{50pt}

  \textbf{\Large Trajectory Optimisation of a Partially-Reusable Rocket-Scramjet-Rocket Small Satellite Launch System}

  \vspace{25pt}

  {\large Sholto O. Forbes-Spyratos }
    
    \vspace{5pt}

  {\large B.Eng. (Mechanical and Aerospace) (Hons. I) \& B.Sc. (Physics)}


  \vspace{60pt}
                                                                                                                                                                                                                                                                                                                                                                                                                                                                                                                                                                                                                                                                                                                                                                                                                                                                                                                                                                                                                                                 
  \vfill

  {\large A thesis submitted for the degree of Doctor of Philosophy at\\ The University of Queensland in 2018}

  \vspace{20pt}

  School of Mechanical Engineering
  
  \textit{Centre for Hypersonics}

  \vfill
\end{center}

\cleardoublepage

\section*{Abstract}

 The rapidly expanding small satellite industry is driving a need for dedicated, cost effective, small satellite launchers. The small satellite launchers currently under development are mostly designed to be disposable,  contrary to the design trends of larger launch systems, which are moving towards reusability. These contradictory design philosophies are due to the unfavourable scaling of the systems necessary for reusability, which are not cost effective for small sized launch systems. 
 A possible way to introduce effective reusability to small scale launch systems is to incorporate airbreathing engines within the launch system design. Scramjets are particularly suitable for launch system integration, due to their high Mach number operation and wide operational regime. 
 Because of these advantages, work is ongoing at The University of Queensland to develop a small satellite launch system incorporating a scramjet accelerator. This launcher is a rocket-scramjet-rocket, three stage system, designed to be partially reusable. 
 The trajectory simulations of this launch system have previously been designed assuming that flying the scramjet-powered stage at its maximum dynamic pressure will maximise the efficiency of the scramjet engines, and that this will maximise the efficiency of the launch system. 
 However, there are complex trade-offs between each stage which must be accounted for within a partially-airbreathing launch system, due to the complex trade-offs between the rocket and airbreathing stages. The rocket stages perform significantly better at high altitudes, where the efficiency loss due to the drag and back pressure from the atmosphere are diminished. Conversely, the airbreathing stage requires high density air to operate effectively, and will generally perform better at low altitudes. 
 This work develops an optimal trajectory profile for a rocket-scramjet-rocket, three stage launch system, determining the trade-offs between the stages necessary to maximise the payload-to-orbit capabilities of the launch system. 
 
 Significant work has previously been carried out on the design and shape optimisation of the scramjet-powered stage, which is designated the SPARTAN. 
 However, the first stage rocket has not been previously designed, and the third stage rocket has previously been designed to use an Aerojet Rocketdyne RL-10-3A motor, which is a costly, pump-fed motor. 
 A first stage rocket is developed, based on a Falcon-1e scaled down lengthwise to 8.5m. This is found to be the appropriate size to deliver the SPARTAN to Mach 5, the minimum operable conditions of the scramjet engines. The third stage rocket is redesigned, to be powered by a SpaceX Kestrel upper stage motor, which is significantly more cost effective than the RL-10-3A used in previous studies. 
 The aerodynamics of the first stage and the SPARTAN are simulated across their operable regimes, including control surface deflection, using inviscid CFD. Aerodynamic databases are calculated for both engine-on and engine-off aerodynamic for the SPARTAN, along with a viscous correction, producing accurate aerodynamic databases.  
 The aerodynamics of the third stage are modelled using Missile Datcom.
 A package is created to calculate the maximum payload-to-orbit trajectory of the rocket-scramjet-rocket launch system in six degrees of freedom, designated LODESTAR.
 In order to determine the maximum payload-to-orbit trajectory shape, optimal control theory is used. For this, LODESTAR utilises the GPOPS-2 a pseudospectral method optimal control software. 
 
 The launch trajectory is initially simulated without the fly-back of the SPARTAN, assuming that the SPARTAN lands at some position downrange. A launch trajectory is first simulated in which the SPARTAN flies at maximum dynamic pressure, for comparison. In order for the SPARTAN to fly at its maximum dynamic pressure for the duration of its trajectory, the first stage must pitch rapidly, flying at negative angle of attack. The SPARTAN then flies at close to horizontal flight, releasing the third stage at a low trajectory angle. It is found that the maximum payload-to-orbit trajectory of the launch system differs significantly from the trajectory in which the SPARTAN flies at maximum dynamic pressure. 
 The SPARTAN is found to deviate from its maximum dynamic rpessure at three points along its trajectory.
 Firstly, the SPARTAN is separated from the first stage rocket at a significantly higher altitude and trajectory angle than the trajectory with the SPARTAN flying at maximum dynamic pressure, causing the altitude of the SPARTAN to increase at the beginning of its trajectory. This reduces the efficiency of the SPARTAN, but increases the efficiency of the first stage rocket, for an increase in the overall efficiency of the system. 
  Additionally, in the middle of its trajectory altitude raising manoeuvre. This altitude raising manoeuvre is performed in a region where the specific impulse of the scramjet engines is relatively homogeneous with varied flight conditions, and results in a very small performance increase. Finally, the SPARTAN performs a pull-up manoeuvre at the end of its trajectory, increasing the altitude and trajectory angle at which the third stage is separated. This pull-up manoeuvre decreases the performance of the SPARTAN, in exchange for greatly increasing the efficiency of the third stage rocket. Overall, flying an optimal trajectory increases the payload-to-orbit of the system launching to sun synchronous orbit to XX, and increases the exergy efficiency of the system by XX.  
 
 The trajectory of the launch system with the fly-back of the SPARTAN has not previously been simulated. Additionally, the separation and fly-back of a hypersonic accelerator, from a relatively low altitude and a Mach number close to 9, has not previously been demonstrated. 
 The fly-back of the SPARTAN is included within the trajectory optimisation, and a maximum payload-to-orbit flight path is simulated. 
 It is found that the SPARTAN must ignite its scramjet engines during its return flight, causing the fly-back to become an important consideration in the optimal trajectory shape. When fly-back is included, the first stage pitches towards the east, though the final orbital inclination is a polar sun synchronous orbit.  
After an easterly release, the SPARTAN banks heavily, and performs a heading angle change manoeuvre throughout its acceleration. This manoeuvre results in a reduction in the ground distance necessary to be covered during the return flight, decreasing the amount of fuel used during fly-back. However, higher angles of attack are required throughout the acceleration of the SPARTAN, decreasing the performance of the SPARTAN by XX compared to the optimal trajectory with no fly-back. 
The fly-back is found to exhibit multiple 'skipping' manoeuvres. These skipping manoeuvres serve to increase the glide range of the SPARTAN, minimising the fuel necessary during return flight. Additionally, the scramjet engines are powered on at the trough of the first three skips, which are sized to allow the scramjet engines to ignite at the points of the highest possible specific impulse. In total, XX\% of the SPARTAN's fuel mass is used during the return flight, and the launch system is able to deliver XXkg of payload to sun synchronous orbit.

The design of the launch system is still undergoing significant modifications and improvements.
For this reason, a sensitivity study is conducted, in order to quantify the sensitivity of the launch system to variations in key design parameters, and investigate the behaviour of the maximum payload-to-orbit trajectory as the characteristics of the launch system are modified. A sensitivity study is conducted for simulations both with, and without, SPARTAN fly-back. 
It is found that in all cases, as the 'useful' energy available to the SPARTAN is increased, the trade-off between the efficiency of the SPARTAN and the third stage rocket shifts towards the SPARTAN, and vice-versa.
The sensitivities of couples design parameters are compared, to quantify their relative impact on the performance of the launch system. Of particular interest is maximum dynamic pressure of the SPARTAN, as the structural and thermal protection mass necessary on board the SPARTAN is closely related to the maximum dynamic pressure. It is found that if a reduction in the maximum dynamic pressure of the SPARTAN by -1kPa reduces the mass of the SPARTAN by greater than -26.5kg (-28.4kg with SPARTAN fly-back), then the performance of the launch system will improve. 

 

\clearpage
\section*{Declaration by author}

  This thesis is composed of my original work, and contains no material previously published or written by another person except where due reference has been made in the text. I have clearly stated the contribution by others to jointly-authored works that I have included in my thesis.

  I have clearly stated the contribution of others to my thesis as a whole, including statistical assistance, survey design, data analysis, significant technical procedures, professional editorial advice, and any other original research work used or reported in my thesis. The content of my thesis is the result of work I have carried out since the commencement of my research higher degree candidature and does not include a substantial part of work that has been submitted to qualify for the award of any other degree or diploma in any university or other tertiary institution. I have clearly stated which parts of my thesis, if any, have been submitted to qualify for another award.

  I acknowledge that an electronic copy of my thesis must be lodged with the University Library and, subject to the policy and procedures of The University of Queensland, the thesis be made available for research and study in accordance with the Copyright Act 1968 unless a period of embargo has been approved by the Dean of the Graduate School.

  I acknowledge that copyright of all material contained in my thesis resides with the copyright holder(s) of that material. Where appropriate I have obtained copyright permission from the copyright holder to reproduce material in this thesis.

\clearpage
\section*{Publications during candidature}

\subsection*{Journal papers}

\noindent\AtNextCite{\defcounter{maxnames}{99}}\fullcite{ForbesSpyratos2018}\\


\subsection*{Conference papers}

\noindent\AtNextCite{\defcounter{maxnames}{99}}\fullcite{ForbesSpyratos2017}\\


\section*{Publications included in this thesis}

This thesis comprises partly of publications, as allowed by University of Queensland Policy PPL 4.60.07. The papers that have been included have all been published in peer reviewed journals at the time of submission. 


\textcolor{red}{make this table more compact}


\vspace{\baselineskip}
\AtNextCite{\defcounter{maxnames}{99}}\fullcite{ForbesSpyratos2018}

\begin{center}
  \begin{tabular}{ll}
    \toprule
    Contributor   & Contribution \\
    \midrule
    Sholto O. Forbes-Spyratos             
                                  & Performed simulations (\%)\\
                                  & Analysis of results (\%)\\
                                  & Wrote and edited paper (\%)\\
    \midrule
    Ingo H. Jahn            
                                  & Analysis of results (\%)\\
                                  
        \midrule
        Michael P. Kearney            
								      & Analysis of results (\%)\\                          

    \midrule
    Michael K. Smart              
                                  & Analysis of results (\%)\\
                                  & Wrote and edited paper (\%)\\
    \bottomrule
  \end{tabular}
\end{center}



\section*{Contributions by Others to the Thesis}

Except for the contributions by others to publications above, there are no contributions by others to this thesis.

\section*{Statement of Parts of the Thesis Submitted to Qualify for the Award of Another Degree}

None.

\clearpage

\section*{Acknowledgements} 
  

  \vfill


\clearpage
\subsection*{Keywords}
  keyword, keyword, keyword, keyword, keyword

\subsection*{Australian and New Zealand Standard Research Classification (ANZSRC)}

  ANZSRC code: 090107 Hypersonic Propulsion and Hypersonic Aerodynamics, 20\% \newline
 ANZSRC code: 090106 Flight Dynamics, 20\% \newline
 ANZSRC code: 090108, Satellite, Space Vehicle and Missile Design and Testing, 20\% \newline
ANZSRC code: 090104 Aircraft Performance and Flight Control Systems , 20\% \newline
ANZSRC code: 010303 Optimisation , 20\% \newline

\subsection*{Fields of Research (FoR) Classification}

  FoR code: 0901, Aerospace Engineering, 100\%

\tableofcontents

\listoffigures
\addcontentsline{toc}{chapter}{List of figures}
\listoftables
\addcontentsline{toc}{chapter}{List of tables}
\printnomenclature